\chapter{Les cristaux ioniques}\index{cristal!ionique|(}
\section{Modèle de description}
Un cristal ionique est un assemblage électriquement neutre d'ions positifs (cations)
et d'ions négatifs (anions) : d'où un double réseau. Le modèle des sphères dures sera
de nouveau utilisé pour les ions.\index{réseau!double}
Les anions en général plus volumineux que les cations constituent un <<cristal hôte>>.
\index{hôte, cristal|see{réseau!hôte}}
Les cations s’insèrent dans les sites de ce cristal en respectant les contraintes 
géométriques. Les ions de signe opposé sont en contact, de sorte que l'interaction attractive
soit la plus forte possible. Les anions constituant le cristal hôte ne sont alors 
pas tangents. La proportion anions/cations dans la structure cristalline est régie 
par la règle de la neutralité électrique.\\
Dans tout le chapitre, on utilisera les paramètres :
\begin{description}
    \item[a] paramètre de maille;
    \item[$r_+$] rayon du cation;
    \item[$r_-$] rayon de l'anion;
    \item[$x$] rapport $\frac{r_+}{r_-}$;
    \item[$Z_+$] nombre de cations par maille;
    \item[$Z_-$] nombre d'anions par maille.
\end{description}

\section{Structure NaCl: coordinence 6\---6}\index{NaCl|(}
De nombreux oxydes métalliques de formule brute MO cristallisent dans cette
structure, l'ion O$^{2-}$ étant relativement petit ($r(\text{O}^{2-}) = 126$pm).
\index{oxyde métallique}

\begin{figure}
    \centering
    \includegraphics[width=7cm]{chap3/nacl_cristal.png}
    \caption{Vue éclatée du cristal de NaCl}\label{fig:3_nacl_cristal}
\end{figure}
\subsection{Description de la structure}
Les anions occupent les n\oe uds d'un réseau de translation cubique à faces
centrées et les cations occupent les cavité octaédriques de la structure,
au centre du cube et au milieu des arêtes.\\

\begin{figure}
    \centering
    \subfloat[Vue compacte]
    {\includegraphics[width=4cm]{chap3/nacl_maille_compacte.png}
    \label{sfig:3_nacl_maille_compacte}}
    \qquad
    \subfloat[Vue éclatée]
    {\includegraphics[width=4cm]{chap3/nacl_maille_eclatee.png}
    \label{sfig:3_nacl_maille_eclatee}}
    \caption{Maille de NaCl}\label{fig:3_nacl_struct}
\end{figure}


\subsection{Nombre de motifs par maille}
Voir figure~\ref{sfig:3_nacl_maille_eclatee}.
\begin{itemize}
    \item 8 anions sommets
    \item 6 anions centres de face
\end{itemize}
D'où :
\begin{equation}
    Z_- = 8 \times \frac{1}{8} + 6 \times \frac{1}{2} = 4
\end{equation}


Les cations sont dans les cavités octaédriques :
\begin{itemize}
    \item 12 cations sur les arêtes
    \item un cation au centre du cube
\end{itemize}
Donc :
\begin{equation}
    Z_+ = 12 \times \frac{1}{4} + 1 \times 1 = 4
\end{equation}
Il y a quatre motifs NaCl par maille, la neutralité
électrique est respectée.


\subsection{Coordinence}
Chaque cation est par construction entouré de manière octaédrique par six anions
de signe opposé. Comme les réseaux cfc des anions et des cations sont équivalents, il
s'agit bien d’une coordinence 6/6. Cf figure~\ref{fig:3_nacl_voisins}.
\begin{figure}
    \centering
    \includegraphics[width=3cm]{chap3/nacl_voisins_2.png}
    \caption{Voisinage d'un ion dans le cristal de NaCl}
    \label{fig:3_nacl_voisins}
\end{figure}


\subsection{Paramètre géométrique}
\begin{figure}
    \centering
    \includegraphics[width=7cm]{chap3/nacl_prm_geo.png}
    \caption{Coupes d'une maille de NaCl}\label{fig:3_nacl_maille_coupe}
\end{figure}
Les cations sont tangents aux anions le long d'une arête (cf figure~\ref{fig:3_nacl_maille_coupe}).
Cela engendre:
\begin{equation}
    2r_+ + 2r_- = a
    \label{eq:3_nacl_ar}
\end{equation}


\subsubsection{Condition limite de stabilité}
Le cas critique est atteint lorsque les anions sont tangents sur la diagonale de face
de leur cristal (voir figure~\ref{fig:3_nacl_maille_coupe}), ce qui donne:
\begin{equation}
    4r_- = a\sqrt{2}
    \label{eq:3_limite_ar-}
\end{equation}
D'où avec~\ref{eq:3_nacl_ar}
\begin{equation}
    x_\text{lim} = \sqrt{2} - 1
\end{equation}
D'où la condition de stabilité :
\begin{equation}
    \boxed{x \geq \sqrt{2} - 1}
    \label{eq:3_stab_x}
\end{equation}

Dans le cas du NaCl : $r_+ = 98$pm, $r_- = 181$pm : $x = \np{0,54}$
\index{NaCl|)}


\section{Structure blende: coordinence 4\---4}\index{blende|(}\index{ZnS|see{blende}}
\subsection{Description de la structure}
Les anions, ici les ions sulfures, occupent les n\oe uds d'un réseau cfc et les cations,
ici les ions zinc(II), occupent la moitié des cavités tétraédriques de l’assemblage
anionique, soit alternativement quatre des centres des huit petits cubes en lesquels on peut
découper le cube de base de l’assemblage anionique. Voir figures~\ref{sfig:3_zns_maille_compacte}
et~\ref{sfig:3_zns_maille_eclatee}
\begin{figure}
    \centering
    \subfloat[Vue compacte]
    {\includegraphics[width=4cm]{chap3/zns_compacte.png}
    \label{sfig:3_zns_maille_compacte}}
    \qquad
    \subfloat[Vue éclatée]
    {\includegraphics[width=4cm]{chap3/zns_eclatee.png}
    \label{sfig:3_zns_maille_eclatee}}
    \qquad
    \subfloat[Tangence dans la maille de blende]
    {\includegraphics[width=4cm]{chap3/zns_tg.png}
    \label{sfig:3_zns_tg}}
    \caption{Représentations d'une maille de blende}\label{fig:3_zns}
\end{figure}


\subsection{Nombre de motifs par maille}
Voir figure~\ref{sfig:3_zns_maille_eclatee} :
\begin{itemize}
    \item 8 anions sommets
    \item 6 anions centes de face
    \item 4 cations dans la moitié des sites tétraédriques
\end{itemize}
\begin{gather*}
    Z_- = 8 \times \frac{1}{8} + 6 \times \frac{1}{2} = 4\\
    Z_+ = 4 \times 1 = 4
\end{gather*}
\begin{equation}
    Z = 4
\end{equation}
La neutralité électrique est respectée.

\subsection{Coordinence}
Voir figure~\ref{sfig:3_zns_tg} :\\
\begin{gather}
    \text{S}_{2^-}/\text{S}_{2^-} = \text{Zn}_{2^+}/\text{Zn}_{2^+} = 12\\
    \text{Zn}_{2^+}/\text{S}_{2^-} = 4
\end{gather}


\subsection{Paramètre géométrique}
Un cation est tangent avec un anion sur la diagonale du cube d'arête $\frac{a}{2}$
(cf figure~\ref{sfig:3_zns_tg}),
La diagonale d'un cube d'arête $\frac{a}{2}$ vaut $D = a\frac{\sqrt{3}}{2}$.\\
Or :
\begin{equation}
    r_- + r_+ = \frac{D}{2}
\end{equation}
On obtient finalement en remplaçant:
\begin{equation}
    \boxed{a = \frac{4\sqrt{3}}{3}(r_- + r_+)}
    \label{eq:3_zns_ar+r-}
\end{equation}


\subsection{Condition de stabilité}
Le cas critique est atteint lorsque les anions sont tangents
selon une diagonale d'une face de la maille, c'est-à-dire,
lorsque :
\begin{equation}
    4r_- = a\sqrt{2}
\end{equation}
On obtient, en utilisant $x$ et en remplaçant $a$ avec~\ref{eq:3_zns_ar+r-}:
\begin{equation}
     \boxed{x \geq \sqrt{\frac{3}{2}} - 1 \approx \np{0,22}}
    \label{eq:3_zns_stab}
\end{equation}
\begin{rem}
    Pour le sulfure de zinc le rapport des rayons ioniques donne
    $x = \frac{83}{190} = \np{0,44}$, résultat incompatible avec les précédents.
    L'explication de cette incohérence tient à une
    modélisation abusive : les ions sulfure, fortement polarisables, sont déformés 
    par la présence de cations Zn$_{2^+}$ qui sont petits, fortements chargés et 
    par conséquent fortement polarisants\index{polarité}.
    \index{sphères dures, modèle des}
\end{rem}
\index{blende|)}



\section{Structure CsCl: coordinence 8\---8}\index{CsCl|(}
Très peu de composés ioniques cristallisent dans cette structure, en particulier 
aucun oxyde binaire métallique. Seuls les halogénures de césium, excepté le fluorure,
\index{halogénures}\index{métal}
cristallisent selon cette structure dans les conditions ordinaires de température et de
pression. Cette structure est néanmoins courante pour les alliages.\index{alliage}


\subsection{Description de la structure}
L'assemblage hôte est un système cubique simple de paramètre $a$. Les anions sont
situés aux sommets du cube et les cations aux centres des cubes anioniques, dans une
cavité à symétrie cubique, voir figure~\ref{fig:3_cscl_struct}.
\begin{figure}
    \centering
    \includegraphics[width=5cm]{chap3/cscl_ecl.png}
    \caption{Structure critalline du CsCl}\label{fig:3_cscl_struct}
\end{figure}


\subsection{Nombre de motifs par maille}
La maille cubique contient en propre un anion et un cation, soit un groupe CsCl.
La neutralité et la st\oe chiométrie sont bien respectées.
\begin{figure}
    \centering
    \subfloat[Réseau hôte]
    {\includegraphics[width=3cm]{chap3/cscl_1.png}
    \label{sfig:3_cscl_resau_hote}}
    \qquad
    \subfloat[Vue compacte]
    {\includegraphics[width=3cm]{chap3/cscl_2.png}
    \label{sfig:3_cscl_compacte}}
    \qquad
    \subfloat[Tangence]
    {\includegraphics[width=3cm]{chap3/cscl_3.png}
    \label{sfig:3_cscl_tg}}
    \caption{Maille de CsCl}\label{fig:3_cscl_maille}
\end{figure}

\subsection{Coordinence}
Chaque cation est entouré de huit anions (voir figure~\ref{fig:3_cscl_maille})
et comme les réseaux anioniques et cationiques sont équivalents, la coordinance
de chaque ions prend la valeur 8.


\subsection{Paramètre géométrique}
Le cation est tangent aux anions selon la grande diagonale du cube. Dans une
situation de contact, on a:
\begin{equation}
    2(r_+ + r_-) = a\sqrt{3}
    \label{eq:3_cscl_ar+r-}
\end{equation}
En utilisant le rapport des rayons, on obtient:
\begin{equation}
    r_- = \frac{a\sqrt{3}}{2(x + 1)}
    \label{eq:3_cscl_r-x}
\end{equation}


\subsection{Condition de stabilité}
La condition limite est atteinte lorsque les anions sont tangents selon les arêtes
du cube, c'est-à-dire:
\begin{equation}
    2r_{-\text{,lim}} = a
\end{equation}
On a donc :
\begin{equation}
    2r_- \leq a
\end{equation}
En remplaçant $r_-$ avec~\ref{eq:3_cscl_r-x}, on obtient:
\begin{equation}
    \boxed{x \geq \sqrt{3} - 1 \approx \np{0,73}}
\end{equation}

\begin{rem}
    Pour CsCl, le rapport $x = \np{0,92}$ et les résultats 
    expérimentaux sont 
    compatibles avec le modèle. Il en est de même pour CsBr
    ($x = \np{0,86}$) et pour CsI ($x = \np{0,77}$).
\end{rem}
\begin{rem}
    \`A température supérieure à 469\degre C, le chlorure de césium
    prend la structure du NaCl (coodinence 6\---6).\index{NaCl}
\end{rem}
\index{CsCl|)}



\section{Structure fluorine: coordinence 8\---4}\index{fluorine|(}\index{CaF$_2$|see{fluorine}}
La structure fluorine est adoptée par le fluorure de calcium ou fluorine de formule
brute CaF$_2$. Elle se rencontre pour d'autres fluorures métalliques comme
BaF$_2$, CdF$_2$ ou des oxydes comme l'oxyde de 
thorium ThO$_2$\index{oxyde}\index{fluorure métallique}


\subsection{Description de la structure}
Il est plus simple de considérer le <<sous cristal>> des cations comme structure de base.
Les cations occupent les n\oe uds d'un réseau cfc de côté $a$, très peu compact. Les anions
occupent les sites tétraédriques c'est-à-dire les centres des huit petits cubes.
\begin{figure}
    \centering
    \includegraphics[width=12cm]{chap3/fluorine.png}
    \caption{Structure cristalline de la fluorine}\label{fig:3_fluorine}
\end{figure}


\subsection{Coordinence}
Chaque maille contient quatre groupes CaF$_2$. Chaque anion est donc entouré de
manière tétragonale par quatre cations. La coordinence d'un anion est de 4.
Chaque cation est entouré <<cubiquement>> par huit anions. La coordinence est donc
de 8. La coordinence du cristal est par conséquent 8\--- 4.


\subsection{Paramètre géométrique}
Le contact anion/cation est réalisé le long d'une demi diagonale du cube anionique,
d'où:
\begin{equation}
    r_+ + r_- = \frac{a\sqrt{3}}{4}
    \label{eq:3_caf2_r+r-a}
\end{equation}


\subsection{Condition de stabilité}
Le remplissage critique correspond à:
\begin{equation}
    2r_- = a
    \label{eq:3_caf2_lim_r-a}
\end{equation}
D'où le rapport critique:
\begin{equation}
    x_\text{lim} = \sqrt{3} - 1
    \label{eq:3_caf2_xlim}
\end{equation}
Et donc la condition de stabilité:
\begin{equation}
    \boxed{x \geq \sqrt{3} - 1}
    \label{eq:3_caf2_stabx}
\end{equation}
Pour le fluorure de calcium, $r_+ = 94$pm, $r_- = 133$pm :
$x = \np{0,74}$\index{fluorine|)}


\section{Propriétés physiques}
Les températures de fusion sont donc élevées (801\degre C pour NaCl)
\index{NaCl!fusion, température de} \index{fusion, température de!NaCl}.
Les ions étant fixes à l'état solide, les cristaux ioniques sont des isolants.
\index{conductivité électrique} Les cristaux ioniques sontdures et solubles
\index{solubilité}dans un solvant polaire. Les forces d’attraction électrostatiques
tridimensionnelles entraînent une forte cohésion du cristal ionique donc des 
énergies de cohésion de l’ordre de 100 à 600kJ/mol.
\index{cohésion, énergie de!cristal ionique}
\index{cristal!ionique|)}
