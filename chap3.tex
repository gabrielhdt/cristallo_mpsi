\chapter{Les cristaux ioniques}\index{cristal!ionique}
\section{Modèle de description}
Un cristal ionique est un assemblage électriquement neutre d'ions positifs (cations)
et d'ions négatifs (anions) : d'où un double réseau. Le modèle des sphères dures sera
de nouveau utilisé pour les ions.
Les anions en général plus volumineux que les cations constituent un "cristal hôte".
Les cations s’insèrent dans les sites de ce cristal en respectant les contraintes 
géométriques. Les ions de signe opposé sont en contact, de sorte que l'interaction attractive
soit la plus forte possible. Les anions constituant le cristal hôte ne sont alors 
pas tangents. La proportion anions/cations dans la structure cristalline est régie 
par la règle de la neutralité électrique.

\section{Structure NaCl: coordinence 6\--- 6}\index{NaCl}
De nombreux oxydes métalliques de formule brute MO cristallisent dans cette
structure, l'ion O$^{2-}$ étant relativement petit ($r(\text{O}^{2-} = 126$pm).


\subsection{Description de la structure}
Les anions \index{anions} occupent les n\oe uds d'un réseau de translation cubique à faces
centrées et les cations occupent les cavité octaédriques \index{site!octaédrique} de la structure,
au centre du cube et au milieu des arêtes.\\
On utilisera les paramètres :
\begin{description}
    \item[a] paramètre de maille
    \item[$r_+$] rayon du cation
    \item[$r_-$] rayon de l'anion
    \item[$x$] rapport $\frac{r_+}{r_-}$ 
\end{description}
\begin{figure}
    \centering
    \subfloat[Structure cristalline du NaCl]
    {\includegraphics[width=5cm]{chap3/nacl_cristal.png}
    \label{sfig:3_nacl_cristal}}
    \qquad
    \subfloat[Maille de Nacl]
    {\includegraphics[width=6cm]{chap3/nacl.png}\label{sfig:3_nacl_maille}}
    \caption{Structure cristalline du NaCl}\label{fig:3_nacl_struct}
\end{figure}


\subsection{Nombre de motifs par maille}
\begin{itemize}
    \item 8 anions sommets
    \item 6 anions centres de face
\end{itemize}
D'où :
\begin{equation}
    Z_- = 8 \times \frac{1}{8} + 6 \times \frac{1}{2} = 4
\end{equation}


Les cations sont dans les cavités octaédriques \index{site!octaédrique} :
\begin{itemize}
    \item 12 cations sur les arêtes
    \item un cation au centre du cube
\end{itemize}
Donc :
\begin{equation}
    Z_+ = 12 \times \frac{1}{4} + 1 \times 1 = 4
\end{equation}
Il y a quatre motifs NaCl par maille.


\subsection{Coordinence}
Chaque cation est par construction entouré de manière octaédrique par six anions
de signe opposé. Comme les réseaux cfc des anions et des cations sont équivalents, il
s'agit bien d’une coordinence 6/6.
\begin{figure}
    \centering
    \includegraphics[width=5cm]{chap3/voisins_nacl.png}
    \caption{Plus proches voisins d'un ion dans le cristal de NaCl}
    \label{fig:3_nacl_voisins}
\end{figure}


\subsection{Paramètre géométrique}
\begin{figure}
    \centering
    \includegraphics[width=7cm]{chap3/nacl_prm_geo.png}
    \caption{Coupe d'une maille de NaCl}\label{fig:3_nacl_maille_coupe}
\end{figure}
Les cations sont tangents aux anions le long d'une arête. Cela engendre:
\begin{equation}
    2r_+ + 2r_- = a
    \label{eq:3_nacl_ar}
\end{equation}


\subsubsection{Condition limite de stabilité}
Le cas critique est atteint lorsque les anions sont tangents sur la diagonale de face
de leur cristal (voir figure~\ref{fig:3_nacl_maille_coupe}), ce qui donne:
\begin{equation}
    4r_- = a\sqrt{2}
    \label{eq:3_limite_ar-}
\end{equation}
D'où avec~\ref{eq:3_nacl_ar}
\begin{equation}
    x_\text{lim} = \sqrt{2} - 1
\end{equation}
D'où la condition de stabilité :\index{stabilité!structure NaCl}
\begin{equation}
    \boxed{x \geq \sqrt{2} - 1}
    \label{eq:3_stab_x}
\end{equation}

Dans le cas du NaCl : $r_+ = 98$pm, $r_- = 181$pm : $x = \np{0,54}$



\section{Structure Blende: coordinence 4\--- 4}




























