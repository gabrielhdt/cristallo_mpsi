\chapter{Les cristaux métalliques}\index{cristal!métallique|(}
L'expérience montre que les cristaux métalliques sont parmis les plus
compactes. Leur structure spatiale sera décrite en faisant l'hypothèse
que les motifs sont équivalents à des sphères indéformables, de rayon
$r$ et l'empilement des motifs est le plus compact possible (les
sphères sont au contact).\index{compacité}\index{sphères dures, modèle des}

\section{Les structures cristallines compactes}
\index{coordinence!structure compacte}
Toutes les structures cristallines compactes ont une coordinence de
12, voir figure~\ref{fig:coordinence_cfc}.
\subsection{Structure CFC}\index{CFC}
Trois plans A, B, C (cf figure~\ref{fig:plans_métal}) se répètent
dans l'ordre A,B,C,A,B, \dots constituant une structure cubique à face
centrée, voir figure~\ref{fig:maille_métal_couches}.
La figure~\ref{fig:maille_cfc} offre une vue simplifiée de la maille.
\begin{figure}
    \centering
    \subfloat[Plan A]
    {\includegraphics[width=4cm]{chap2/plan_a.png}}
    \qquad
    \subfloat[Plan B]
    {\includegraphics[width=4cm]{chap2/plan_b.png}}
    \qquad
    \subfloat[Plans A,B et C]
    {\includegraphics[width=3cm]{chap2/plan_c.png}}
    \caption{Plans de la structure cfc}\label{fig:plans_métal}
\end{figure}
\begin{figure}
    \centering
    \subfloat[Vue en perspective]
    {\includegraphics[width=6cm]{chap2/crist_met.png}}
    \qquad
    \subfloat[Vue plane]
    {\includegraphics[width=6cm]{chap2/plans_met.png}}
    \caption{Maille cfc
    en fonction des couches}\label{fig:maille_métal_couches}
\end{figure}
\begin{figure}
    \centering
    \input{pictures/cfc.pdf_tex}
    \caption{Maille d'une structure cfc}\label{fig:maille_cfc}
\end{figure}

\subsection{Structure HC}\index{HC}
Deux plans A, B (cf figure~\ref{fig:plan_ab_hc}) se répètent A,B,A, \dots
formant une structure hexagonale compacte.
\begin{figure}
    \centering
    \subfloat[Vue éclatée]
    {\includegraphics[width=4cm]{chap2/plans_ab_hc_eclatee.png}
    \label{sfig:2_plans_ab_hc_éclatée}}
    \qquad
    \subfloat[Vue compacte]
    {\includegraphics[width=4cm]{chap2/plans_ab_hc_compacte.png}
    \label{sfig:2_plans_ab_hc_compacte}}
    \caption{Plans de la structure hc}\label{fig:plan_ab_hc}
\end{figure}
En se ramenant aux réseaux de Bravais (tableau~\ref{tab:bravais}
page~\pageref{tab:bravais}),
la maille est un volume engendré par la translation d'un losange,
visible sur la figure~\ref{fig:maille_hc}.
\begin{figure}
    \centering
    \subfloat[Base de la maille]
    {\includegraphics[width=4cm]
        {chap2/maille_hc_2d.png}\label{sfig:maille_hc_2d}}
    \qquad
    \subfloat[Vue en perspective de la maille]
    {\includegraphics[width=3cm]
        {chap2/maille_hc_3d.png}\label{sfig:maille_hc_3d}}
    \caption{Vues d'une maille de structure hc}\label{fig:maille_hc}
\end{figure}

\begin{figure}
    \centering
    \subfloat[Structure CFC]
    {\includegraphics[width=4cm]{chap2/plans_cfc.png}
    \label{sfig:2_plans_cfc_couleur}}
    \qquad
    \subfloat[Structure HC]
    {\includegraphics[width=4cm]{chap2/plans_hc.png}
    \label{sfig:2_plans_hc_couleur}}
    \caption{Structures cfc et hc}
\end{figure}
\begin{figure}
    \centering
    \subfloat[Structure CC]
    {\includegraphics[width=3.5cm]{chap2/cristaux_metalliques_cc_ens.png}}
    \qquad
    \subfloat[Structure CFC]
    {\includegraphics[width=3.5cm]{chap2/cristaux_metalliques_cfc_ens.png}}
    \qquad
    \subfloat[Structure HC]
    {\includegraphics[width=3.5cm]{chap2/cristaux_metalliques_hc_ens.png}}
    \caption{Structures métalliques}
\end{figure}
\begin{figure}
    \centering
    \subfloat[CC]
    {\includegraphics[width=1cm]{chap2/cristaux_metalliques_cc_maille.png}}
    \qquad
    \subfloat[CFC]
    {\includegraphics[width=1cm]{chap2/cristaux_metalliques_cfc_maille.png}}
    \qquad
    \subfloat[HC]
    {\includegraphics[width=1cm]{chap2/cristaux_metalliques_hc_maille.png}}
    \caption{Mailles métalliques}
\end{figure}

\section{Structure CFC}\index{CFC|(}
Les paramètres suivants seront utilisés :
\begin{description}
    \item[$a$] taille caractéristique d'une maille;
    \item[$r$] rayon du motif (sphère représentant l'atome ou ion);
    \item[$Z$] nombre de motifs par maille;
    \item[$C$] compacité;
    \item[$\lbrack x \rbrack$] coordinence;
    \item[$V$] volume d'une maille;
    \item[$\rho$] masse volumique.
\end{description}
\begin{figure}
    \centering
    \includegraphics[width=4cm]{chap2/coordinence_cfc.png}
    \caption{Plus proches voisins d'un atome dans la
        structure cfc}\label{fig:coordinence_cfc}
\end{figure}
\subsection{Paramètre géométrique}
\begin{figure}
    \centering
    \includegraphics[width=4cm]{chap2/prm_geo_cfc.png}
    \caption{Représentation d'une maille en fonction de $a$}\label{fig:prm_geo_cfc}
\end{figure}

La diagonale d'une maille mesure $\sqrt{2}a$ (par Pythagore).
Les sphères étant tangentes le long de la diagonale, on obtient :
\begin{gather}
    4r = \sqrt{2}a \\
    \boxed{r = \frac{a}{2^{\sfrac{3}{2}}}} \label{eq:r}
\end{gather}
\subsection{Nombre de motifs par maille}
Voir figure~\ref{fig:maille_cfc}.
\begin{itemize}
    \item Chaque motif à un sommet est partagé entre 8 mailles;
    \item chaque motif sur une face est partagé entre deux mailles.
\end{itemize}
\begin{equation}
    Z = 8 \times \frac{1}{8} + 6 \times \frac{1}{2} = 4
\end{equation}
\subsection{Compacité}
La formule de la compacité~\ref{eq:comp} page~\pageref{eq:comp} donne
\begin{gather}
    C = \frac{4 \frac{4}{3} \pi r^3}{a^3}\\
    \intertext{d'où, en remplaçant $r$ avec~\ref{eq:r},}
    \boxed{C = \frac{\pi}{3\sqrt{2}} \approx \np{0,74}}
\end{gather}

\subsection{Sites}
Compte tenu de la compacité, il existe des sites entre les motifs.
\index{compacité}
\begin{rem}
    Ne seront étudiés que les sites de symétrie importante,
    à savoir les sites octaédriques et tétraédriques.
\end{rem}
\begin{figure}
    \centering
    \input{pictures/ex_sites.pdf_tex}
    \caption[Sites d'une structure cfc]
    {Site octaédrique (rouge) et tétraédrique (bleu)}\label{fig:2_cfc_sites}
\end{figure}
\subsubsection{Sites octaédriques}\index{site!octaédrique}
Les sites octaédriques sont au centre de la maille et au milieu
des arêtes. On a donc $1 + 12 \times \frac{1}{4} = 4$ sites octaédriques.
\begin{figure}
    \centering
    \includegraphics[width=3.5cm]{chap2/sites_o_cfc.png}
    \caption{Placement d'une entité dans un site octaédrique}\label{fig:site_o_cfc}
\end{figure}
On peut calculer la taille maximale des atomes étrangers.
Soit $r_{\text{O,max}}$ le rayon maximal d'un atome étranger.
Les sphères (atomes h\^otes et étrangers) étant tangentes le
long du c\^oté de la maille, on a :\index{réseau!hôte}
\index{étranger, atome}
\begin{gather}
    a = 2r + 2r_{\text{O,max}}\\
    \intertext{d'où, en remplaçant $a$ à l'aide de~\ref{eq:r} :}
    \boxed{r_{\text{O,max}} = (\sqrt{2} - 1)r} \label{eq:rmax}
\end{gather}
\begin{rem}
    Il sera convénient d'utiliser le rapport noté $x$ :
    \begin{equation*}
        x = \frac{r_{\text{O,max}}}{r}\text{.}
    \end{equation*}
    Dans le cas d'une cfc :
    \begin{equation*}
        x = \sqrt{2} - 1\text{.}
    \end{equation*}
\end{rem}

\begin{ex}[Cas du fer $\gamma$]\index{fer}
    Données : $r = 124\text{pm}, M = 56\text{g.mol}^{-1}$
    \begin{compactitem}
        \item Déterminer le paramètre de maille $a$;
        \item Déterminer le rayon maximal des atomes pouvant s'insérer dans un
            site octaédrique $r_{\text{O,max}}$;
        \item Peuvent des atomes d'hydrogène s'insérer ($r_{\text{H}} = 37$pm)?
            \index{hydrogène}
        \item Déterminer la masse volumique $\rho$.
    \end{compactitem}
    L'équation~\ref{eq:r} donne
    $a=2^{\sfrac{3}{2}} \times 124\text{pm} = 351\text{pm}$.
    Pour trouver $r_{\text{O, max}}$ on utilise~\ref{eq:rmax} :
    $r_{\text{O,max}} = (\sqrt{2} - 1)\times 124 = 51\text{pm}$.
    On observe que $r_{\text{H}} \leq r_{\text{O,max}}$, des atomes
    d'hydrogènes peuvent donc se positionner dans les sites
    octaédriques. En revanche, l'insertion d'atomes d'azote ou de
    carbone (rayons $70$ et $77$pm) entraîne une déformation importante
    du métal. Le fer nitrué ou carburé (la fonte) sont nettement
    plus fragiles que le fer pur.
    \index{propriétés mécaniques!fer}\index{fer}
    D'après~\ref{eq:masse_volumique} :
    $\rho = \frac{4 \times 56}{\np{6,02}.10^{23}(351.10^{-12}.10^2)^3} = \np{8,60}\text{g.cm}^{-1}$
\end{ex}

\subsubsection{Sites tétraédriques}\index{site!tétraédrique}
Les centres des sites tétraédriques sont situés aux centres des
huit petits cubes en lesquels on peut découper le cube d’arête
$\frac{a}{2}$.
\paragraph{Calcul de $r_{\text{T,max}}$}
Les sphères sont tangentes le long de la diagonale du cube.
En notant $D$ la diagonale du cube, on a $D = \sqrt{3}a$
(par Pythagore). La diagonale $\hat{D}$ d'un cube d'arête $\frac{a}{2}$
vaut donc $\hat{D} = D/2 = \frac{\sqrt{3}a}{2}$.
Comme le site est au milieu du cube d'arête $\frac{a}{2}$, on a :
\begin{gather}
    2r + 2r_{\text{T,max}} = \hat{D}\\
    \intertext{c'est-à-dire, en remplaçant :}
    \boxed{r_{\text{T,max}} = r\Biggl(\sqrt{\frac{3}{2}} - 1\Biggr)}\label{eq:rmax_tetra}\\
    \intertext{D'où,}
    x = \sqrt{\frac{3}{2}} - 1
\end{gather}
\index{CFC|)}


\section{Structure HC}\index{HC|(}
Les métaux alcalino-terreux et de nombreux métaux de transition
adoptent la structure hexagonale compacte.
\index{métal!alcalino-terreu}\index{métal!transition, de}
\begin{description}
    \item[$a$] taille caractéristique d'une maille;
    \item[$c$] hauteur de la maille;
    \item[$r$] rayon du motif (sphère représentant l'atome ou ion);
    \item[$Z$] nombre de motifs par maille;
    \item[$C$] compacité;
    \item[$\lbrack x \rbrack$] coordinence;
    \item[$V$] volume d'une maille;
    \item[$\rho$] masse volumique.
\end{description}
\subsection{Paramètre géométrique}
D'après la figure~\ref{sfig:maille_hc_2d}, il est évident que :
\begin{equation}
    a = 2r\text{.} \label{eq;a_r_hc}
\end{equation}
La maille n'étant pas cubique, il reste le paramètre géométrique
de hauteur $c$ visible sur la figure~\ref{sfig:maille_hc_3d}.
On se place dans le tétraèdre dont la base est le triangle formé
par 3 motifs de la base de la maille et le sommet est le motif
de la deuxième couche. Ce tétraèdre sera paramétré selon le
schéma~\ref{fig:prm_geo_hc}.
On notera $h$ pour la hauteur, donc $h = \text{SH} = \frac{c}{2}$.
Par Pythagore, on a :
\begin{gather}
    a^2 = \text{PH}^2 + h^2 \label{eq:PH_a_h}\\
    \intertext{Comme H est le centre de gravité du triangle PQR, on a}
    \text{PH} = \frac{2}{3}\text{PN}\label{eq:PG_PN}\\
    \intertext{or, en considérant l'angle formé par (PN) et (PR) :}
    \cos\Bigl( \frac{\pi}{6}\Bigr) = \frac{\text{PN}}{a}\\
    \intertext{D'où, en remplaçant PN à l'aide de~\ref{eq:PG_PN}:}
    \text{PH} = \frac{\sqrt{3}a}{3}\label{eq:PH_a}\\
    \intertext{En remplaçant PH dans~\ref{eq:PH_a_h}, on obtient :}
    h = \sqrt{\frac{2}{3}}a\\
    \intertext{Soit finalement}
    \boxed{c=2\sqrt{\frac{2}{3}}a} \label{eq:c_hc}
\end{gather}
\begin{figure}
    \centering
    \includegraphics[width=5cm]{chap2/prm_geo_hc.png}
    \caption{Tétraèdre dans une maille
        de structure HC}\label{fig:prm_geo_hc}
\end{figure}
\begin{rem}
    Un cristal ayant une structure hexagonale compacte
    peut être caractérisé par le rapport $\frac{c}{a}$.
    Ici, $\frac{c}{a} = 2\sqrt{\frac{2}{3}}\approx \np{1.63}$.
    Pour le magnésium, le rapport vaut
    $\np{1,62}$. Cette valeur est très proche de celle
    obtenue ci-dessus, les atomes sont des sphères
    au contact. Pour le titane,
    $\frac{c}{a}=\np{1,59}$ et pour le cadmium,
    $\frac{c}{a}=\np{1,89}$. Les écarts sont donc
    importants. Dans le cadre d’une structure par empilement,
    les motifs ne sont plus des sphères mais des ellipsoïdes
    aplaties dans le cas du cadmium, étirées dans le cas du titane.
    \index{sphères dures, modèle des}
\end{rem}

\subsection{Nombre de motifs par maille}
\begin{itemize}
    \item Chaque motif placé à un sommet (8) est partagé entre
        8 mailles;
    \item un motif entièrement dans la maille situé à
        $\sfrac{c}{2}$ et aligné avec le centre de gravité d'un
        des deux triangles formés par 3 des motifs de la base de
        la maille.
\end{itemize}
\begin{equation}
    Z = \frac{1}{8} \times 8 + 1 = 2
\end{equation}

\subsection{Compacité}
La formule~\ref{eq:comp} donne:
\begin{gather}
    C = \frac{2 \times \frac{4}{3}\pi r^3}{V}\\
    \intertext{Or d'après~\ref{eq:volume} :}
    V=a^2 \frac{\sqrt{3}}{2}\\
    \intertext{Donc en remplaçant $a$ avec~\ref{eq;a_r_hc} :}
    V=2^3\sqrt{2}r^3 \label{eq:V_hc}\\
    \intertext{D'où :}
    \boxed{C = \frac{\pi}{3\sqrt{2}} \approx \np{0,74}}
    \label{eq:comp_hc}
\end{gather}

\subsection{Sites}
\begin{rem}
    L'étude détaillée des sites de la structure hexagonale
    compacte n'est pas au programme officiel.
\end{rem}
Un cristal ayant une structure hc possède des cavités comparables
à celles mises en évidence dans la structure cfc.
\begin{itemize}
    \item 2 cavités octaédriques;
    \item 4 cavités tétraédriques.
\end{itemize}
Compte tenu de l’analogie structurale qui existe entre les deux
systèmes, on admet que la taille des cavités de la structure
hc prend la même valeur que pour les cavités correspondantes
de la structure cfc. Il y a autant de cavités octaédriques que
de motifs dans la maille, soit deux.
\begin{figure}
    \centering
    \input{pictures/sites_hc.pdf_tex}
    \caption{Sites tétraédriques et octaédriques
        de la structure hc}
\end{figure}\index{HC|)}


\section{Exemples}
\begin{ex}[Structure cfc : le fer $\alpha$]\index{fer}\index{CFC}
    $a = 356$pm,
    $M = \np{55,8}$g.mol$^{-1}$.
    \begin{compactitem}
        \item Exprimer $r$ en fonction de $a$ paramètre
            de la maille;
        \item Déterminer le nombre de motifs par maille;
        \item Déterminer la compacité.
    \end{compactitem}
    D'après~\ref{eq:r} : $r = a\frac{\sqrt{2}}{4} \approx \np{124,9}$pm;
    $Z = 4$;
    d'après~\ref{eq:comp_hc}, $C = \frac{\pi}{3\sqrt{2}} \approx \np{0,74}$.
\end{ex}
\begin{ex}[Structure hc : le magnésium]\index{HC}
    $a = 320$pm,
    $M = \np{24,3}$g.mol$^{-1}$.
    \begin{compactitem}
        \item Déterminer la hauteur $c$ de la maille hexagonale
            compacte;
        \item Déterminer la relation entre $a$ et $r$ le rayon
            d'un atome du métal;
        \item Déterminer la coordinence et le nombre de motifs par
            maille;
        \item Déterminer la compacité.
    \end{compactitem}
    D'après~\ref{eq:c_hc} : $c = \np{552,6}$pm; d'après~\ref{eq;a_r_hc}
    $a = 2r$, $Z = 2$, $[x] = 12$.
\end{ex}



\section{Structure CC}\index{CC|(}
Les métaux alcalins et alcalino-terreux, ainsi que quelques
\index{métal!alcalin}\index{métal!alcalino-terreu}\index{fer}
\index{métal!transition, de}
métaux de transition comme le fer $\alpha$ cristallisent selon la structure cc.
En accord, avec le fait que ces structures sont moins compactes,\index{compacité}
il est à signaler que les métaux correspondants (comme le sodium) sont\index{sodium}
plus fragiles et plus malléables.

\subsection{Construction par empilement}
\begin{description}
    \item[Plan A] pavage en plaçant les sphères aux sommets de carrés
        de côté $a$ avec $a>2r$: les sphères ne sont pas au contact.
        Voir figure~\ref{sfig:plana_cc}.
    \item[Plan B] les sphères se placent dans les creux laissés par 
        les sphères du plan A, leurs centres se projettent sur
        les centres des carrés du plan initial.
        Voir~\ref{sfig:planb_cc_dessus} et~\ref{sfig:planb_cc_cote}.
    \item[Plan C] voir plan A et figure~\ref{sfig:planc_cc}.
\end{description}
\begin{figure}
    \subfloat[Plan A]
    {\includegraphics[width=3cm]
        {chap2/plana_cc.png}\label{sfig:plana_cc}}
    \qquad
    \subfloat[Plan B, vue de dessus]
    {\includegraphics[width=3cm]
        {chap2/planb_cc_1.png}\label{sfig:planb_cc_dessus}}
    \qquad
    \subfloat[Plan B, vue de c\^oté]
    {\includegraphics[width=3cm]
        {chap2/planb_cc_2.png}\label{sfig:planb_cc_cote}}
    \qquad
    \subfloat[2\ieme plan A]
    {\includegraphics[width=3cm]
        {chap2/planc_cc.png}\label{sfig:planc_cc}}
    \caption{Plans de la structure CC}\label{plans_cc}
\end{figure}

\subsection{Coordinence}
Voir la figure~\ref{fig:coordinence_cc}.\\
$[x] = 8$
\begin{figure}
    \centering
    \includegraphics[width=3cm]{chap2/coordinence_cc.png}
    \caption{Maille CC}\label{fig:coordinence_cc}
\end{figure}

\subsection{Nombre de motifs par maille}
Voir figure~\ref{fig:coordinence_cc}.
\begin{itemize}
    \item Les 8 motifs aux 8 sommets sont partagés entre 8 mailles;
    \item Le motif au milieu du cube n'est pas partagé.
\end{itemize}
\begin{equation}
    Z = \frac{1}{8} \times 8 + 1 = 2
\end{equation}

\subsection{Compacité}
\begin{gather}
    C = \frac{2\times \frac{4}{3} \pi r^3}{a^3}\\
    \intertext{D'où,}
    C = \pi \frac{\sqrt{3}}{8} \approx \np{0,68}
\end{gather}
\begin{rem}
    La compacité de la structure cc est inférieure à celle
    des structures hc et cfc.
\end{rem}
\begin{ex}[Na]\index{Na|see{sodium}}\index{sodium}
    $r = 190$pm, $M = \np{23,0}$g.mol$^{-1}$.
\end{ex}

\section{Propriétés physiques}
Un élément possédant une faible électronégativité peut former
\index{électronégativité}
à l’état solide un cristal métallique dont la cohésion est
assurée par la liaison métallique.
Dans un modèle simple, on peut considérer que les électrons
de valence des atomes sont délocalisés sur l’ensemble du
\index{valence}
cristal, interagissant avec les cations métalliques situés
aux n\oe uds du réseau (modèle du gaz d’électrons)\index{gaz d'électrons}.
Les cations sont assimilés à des sphères dures
indéformables. La structure est imposée par les ions positifs.
La liaison métallique est forte (énergie de 100 à 800 kJ.mol$^{-1}$)
\index{cohésion, énergie de!CC}
et non directionnelle. Elle est à l’origine de certaines
\index{directionnelle, liaison}
propriétés macroscopiques des métaux :
\begin{itemize}
    \item température de fusion élevée; \index{fusion, température de!CC}
        \index{CC!fusion, température de}
    \item conductivités thermique et électrique élevées;
        \index{conductivité électrique}
    \item propriétés mécaniques de dureté\footnote{résistance à
        la pénétration}, de malléabilité\footnote{aptitude à la
        déformation sans se rompre} et de
        ductilité.\footnote{aptitude au laminage et au filage}
        \index{propriétés mécaniques!métaux}
        \index{solidité|see{propriétés mécaniques!métaux}}
        \index{dureté|see{propriétés mécaniques!métaux}}
        \index{malléabilité|see{propriétés mécaniques!métaux}}
        \index{ductilité|see{propriétés mécaniques!métaux}}
    \item propriétés optiques d'opacité et de pouvoir réflecteur.
        \index{propriétés optiques}
\end{itemize}

\subsection*{Alliages}\index{alliage}
Les cristaux métalliques peuvent conduire à plusieurs types
d'alliage :
\begin{itemize}
    \item lorsque les sites intersticiels de la structure sont
        occupés par des atomes de taille adéquate;
    \item lorsque certains atomes de la structure sont remplacés
        par des atomes métalliques de nature différente.
\end{itemize}
\begin{ex}[Alliages de fer]\index{fer!alliages}
    \begin{compactdesc}
        \item[Fonte] fer et carbone, carbone à plus de $\np{2,1}$\%
            et jusqu'à $\np{6,7}$\% en masse;
        \item[Acier] fer et carbone (moins de $\np{2,1}$\% en masse
            de carbone) et traces éventuelles de nickel, chrome,
            molybdène en faible pourcentage ($<4$\%).
    \end{compactdesc}
\end{ex}
\begin{ex}[Alliages de cuivre]\index{cuivre!alliages}
    \begin{compactitem}
        \item Bronze : cuivre et étain (l'airin est l'ancien nom
            du bronze);
        \item Laiton : cuivre et zinc.
    \end{compactitem}
\end{ex}
\index{CC|)}
\index{cristal!métallique|)}
