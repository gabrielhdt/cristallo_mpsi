\chapter{L'état solide cristallin}
\section{Les différents états solides}
\subsection{Cristal parfait}
C'est le modèle idéal du solide cristallisé, il est constitué d'un arrangement
périodique d’entités (atomes, ions, ou molécules) dans les trois directions
de l’espace et à l’infini.
\begin{rem}
    Le cristal réel s'écarte de ce modèle
\end{rem}

\subsection{Cristal amorphe}
Un cristal est dit amorphe si l'ordre est local, et le désordre existe
à longue distance.
\begin{ex}[le verre]
    les verres sont le plus souvent obtenus par refroidissement d’un liquide
    n’ayant pas pu cristalliser; ils apparaissent comme un liquide figé 
    d’énorme viscosité. Cette forme de la matière est appelé état vitreux.
\end{ex}

\subsection{Cristal réel}\index{cristal!réel}
Le cristal réel présente souvent des défauts (absence d'une entité, déplacement
d'une entité par rapport à sa position parfaite \dots). Il n'est pas toujours 
constitué d’un unique cristal mais de plusieurs monocristaux juxtaposés
les uns aux autres, on l’appelle alors cristal polycristallin
(voir figure~\ref{fig:quartz_rutile}).
\begin{figure}
    \centering
    \includegraphics[width=10cm]{Quartz_a_rutile.jpg}
    \caption{Quartz à rutile}\label{fig:quartz_rutile}
\end{figure}
\begin{rem}
    Ce cours ne développe que la structure du cristal parfait, c'est-à-dire un
    état \emph{cristallin parfaitement ordonné} à très grande distance
    (idéalement à l'infini).
\end{rem}


\section{Modèle du cristal parfait}
\subsection{Motif}
Un cristal parfait résulte de la répétition périodique à l’infini, dans
les trois dimensions de l’espace, d’une entité élémentaire (atome, ion
ou molécule voire macromolécule) appelée \emph{motif}. Voir la
table~\ref{tab:1_motifs_ex} pour des exemples de motifs.
\begin{defi}
    Le motif d'un cristal est la plus petite entité (atome ou groupe d'atomes)
    qui se répète périodiquement.
\end{defi}
\begin{table}
    \centering
    \begin{tabular}{|l|c|c|c|c|}
        \hline Type de cristal & Métallique & Covalent & Moléculaire
        & Ionique \\
        \hline Motif & Atome & Atome & Molécule & Ions \\
        Exemple & Cu & C & H$_2$O & Na$^+$,Cl$^-$ \\ \hline
    \end{tabular}
    \caption{Exemples de motifs}\label{tab:1_motifs_ex}
\end{table}

\subsection{Réseau}
La description d'un cristal requiert un support géométrique, c’est-à-dire un 
ensemble de points où on place le motif.
\begin{defi}
    Un réseau est un outil géométrique permettant de décrire l’arrangement
    périodique tridimensionnel de points appelés des nœuds.
\end{defi}
\begin{figure}
    \centering
    \includegraphics[width=6cm]{reseau.png}
    \caption{Exemples de réseaux}
\end{figure}
Un cristal est l'association d'un réseau et d'un motif placé aux noeuds du
réseau.

\subsection{Maille}
Les noeuds du réseau se déduisent les uns des autres par des opérations
de translation, combinaison linéaire de trois vecteurs non coplanaires
(usuellement notés $\vec{a}, \vec{b}, \vec{c}$). Ces
trois vecteurs définissent une structure parallélépipédique appelée
\emph{maille}.
\begin{defi}
    Une maille est une unité de base parallélépipédique à partir de laquelle
    on peut décrire tout le cristal en lui faisant subir différentes
    translations suivant les directions des trois axes cristallins.
\end{defi}

\subsection{Les sept systèmes cristallins}
Les réseaux cristallins peuvent être décrits à partir de \emph{7 mailles
élémentaires} qui définissent \emph{7 systèmes cristallins}.
Selon que la maille est simple (un seul motif par maille) ou multiple, et à
partir de ces 7 systèmes cristallins, on définit
\emph{14 réseaux de \bsc{Bravais}} (cf table~\ref{tab:bravais}). On distingue
parmis les systèmes différents modes cristallins :
\begin{description}
    \item[P]  mode primitif, n\oe uds de réseaux seulement aux sommets de la 
        maille;
    \item[I] un n\oe ud de réseau au centre du volume de la maille;
    \item[C] un n\oe ud de réseau au centre de la face C de la maille, définie par 
        les axes $\vec{a}$ et $\vec{b}$;
    \item[F] un n\oe ud de réseau au centre de chaque face de la maille.
\end{description}
\begin{rem}
    Il existe aussi les modes de réseaux A, B et R, mais ils ne seront pas étudiés
    dans ce cours (les modes A et B sont semblables au mode C).
\end{rem}
\begin{rem}
    Les structures cubiques F seront appelées \emph{CFC} pour \emph{cubique à
    faces centrées}. Les structures hexagonales P seront appelées \emph{HC}
    pour \emph{hexagonal compacte}.
\end{rem}

\begin{table}
    \footnotesize
    \begin{tabulary}{\linewidth}{|p{2.5cm}|c|c|c|c|}
        \hline \bf Système & Type P & Type C & Type I & Type F \\\hline
        Cubique, \mbox{$a = b = c$}
        \mbox{$\alpha = \beta = \gamma = \sfrac{\pi}{2}$} &
        \includegraphics[width=2.2cm]{bravais_cp.png} & & 
        \includegraphics[width=2.2cm]{bravais_ci.png} &
        \includegraphics[width=2.2cm]{bravais_cf.png} 
        \\
        \hline Rhomboédrique $a = b = c$,
        \mbox{$\alpha = \beta = \gamma \neq \sfrac{\pi}{2}$}&
        \includegraphics[width=2.2cm]{bravais_rp.png} & & &
        \\
        \hline Hexagonal $a = b \neq c$
        \mbox{$\alpha = \beta = \sfrac{\pi}{2}$ $\gamma = \sfrac{2\pi}{3}$} &
        \includegraphics[width=2.2cm]{bravais_hp.png} & & &
        \\
        \hline Quadratique $a = b \neq c$
        \mbox{$\alpha = \beta = \gamma = \sfrac{\pi}{2}$} &
        \includegraphics[width=2.2cm]{bravais_qp.png} & &
        \includegraphics[width=2.2cm]{bravais_qi.png} &
        \\
        \hline Orthorombique $a \neq b \neq c$
        \mbox{$\alpha = \beta = \gamma = \sfrac{\pi}{2}$} &
        \includegraphics[width=2.2cm]{bravais_op.png} &
        \includegraphics[width=2.2cm]{bravais_oc.png} &
        \includegraphics[width=2.2cm]{bravais_oi.png} &
        \includegraphics[width=2.2cm]{bravais_of.png}
        \\
        \hline Monoclinique $a \neq b \neq c$
        \mbox{$\alpha=\gamma=\sfrac{\pi}{2} \neq \beta$}
        &
        \includegraphics[width=2.2cm]{bravais_mp.png} &
        \includegraphics[width=2.2cm]{bravais_mc.png} & &
        \\
        \hline Triclinique $a \neq b \neq c$
        \mbox{$\alpha \neq \beta \neq \gamma \neq \sfrac{\pi}{2}$} &
        \includegraphics[width=2.2cm]{bravais_tp.png} & & &
        \\
        \hline
    \end{tabulary}
    \caption{Les 14 réseaux de \bsc{Bravais}}\label{tab:bravais}
\end{table}

\subsection{Caractéristiques des réseaux cristallins}
\subsubsection{Volume d'une maille}
\begin{prop}
    \begin{equation}
        V = (\vec{a} \wedge \vec{b}) \centerdot \vec{c}
        \label{eq:volume}
    \end{equation}
\end{prop}

\subsubsection{Modèle des sphères dures}
Les atomes, ions ou molécules constituant le solide cristallin parfait seront
supposés être des sphères indéformables et compactes.

\subsection{Famille cristalline}
Ensemble des matériaux ayant même structure et présentant une forte analogie des
propriétés chimiques et physiques.

\subsubsection{Nombre de motifs par maille}
Une maille est dite \emph{unitaire} si elle ne comporte qu'un seul motif,
\emph{multiple} si elle en comporte plusieurs.
\paragraph{Détermination du nombre de motifs par maille $Z$}
\begin{itemize}
    \item un élément extérieur à la maille compte pour $0$
    \item un élément intérieur à la maille compte pour $1$
    \item un élément au sommet de la maille compte pour $\frac{1}{8}$
    \item un élément sur une arête compte pour $\frac{1}{4}$
    \item un élément sur une face compte pour $\frac{1}{2}$
\end{itemize}

\subsubsection{Coordinence}
\begin{defi}[Coordinence de l'atome ou ion $A$ par rapport à $V$]
    nombre $x$ de plus proches voisins $V$ de $A$ noté $A/V = [x]$
\end{defi}

\subsubsection{Masse volumique}
\begin{defi}[Masse volumique ramenée à une maille]
    Rapport de la masse d'une maille et de son volume :
    \begin{equation}
        \rho = \frac{ZM}{\mathcal{N}_\text{A}V}\label{eq:masse_volumique}
    \end{equation}
    \begin{itemize}
        \item $Z$ nombre de motifs par maille
        \item $M$ masse molaire du motif
        \item $\mathcal{N}_\text{A}$ nombre d'Avogadro
        \item $V$ volume de la maille
    \end{itemize}
\end{defi}

\subsubsection{Compacité $C$}
\begin{defi}
    Nombre sans dimension qui mesure le taux d'occupation réel de l'espace
    par les atomes ou ions assimilés à des sphères.\\
    Elle est toujour comprise entre 0 et 1
    \begin{equation}
        C = \frac{4}{3}\frac{N \pi r^3}{V} \label{eq:comp}
    \end{equation}
    \begin{itemize}
        \item $N$ nombre de motifs
        \item $r$ rayon des atomes ou ions
        \item $V$ volume de la maille
    \end{itemize}
\end{defi}

\subsection{Sites}
Tout réseau cristallin de sphères identiques $S$ de rayon $R$ comporte
nécessairement des portions d'espace non occupées puisque $C < 1$; ils
constituent des sites. Le réseau constitué des sphères $S$ s'appelle le
\emph{réseau h\^ote}.
\begin{defi}[Site intersticiel]
    Centre de polyèdre de coordination, régulier ou non, ayant pour sommets
    les centres des premiers voisins du réseau h\^ote
\end{defi}
\begin{ex}[cf figure~\ref{fig:ex_sites}]
    \begin{compactdesc}
        \item[Site cubique C] le polyèdre de coordination est un cube :
            $C/S = [8]$
        \item[Site octaédrique O] le polyèdre de coordination est un
            octaèdre à 6 sommets et 8 faces $O/S = [6]$
        \item[Site tétraédrique T] le polyèdre de coordination est un
            tétraèdre à 4 sommets et 4 faces $T/S = [4]$
    \end{compactdesc}
\end{ex}
\begin{figure}
    \centering
    \input{pictures/ex_sites.pdf_tex}
    \caption[Sites d'une structure cfc]
    {Site octaédrique (rouge) et tétraédrique (bleu)}\label{fig:ex_sites}
\end{figure}


\section{Les différents modèles de cristaux}
\begin{description}
    \item[Cristaux moléculaires] Ce sont des cristaux ayant pour motif une
        molécule. Ils présentent des intéractions électriques
    \item[Cristaux métalliques] Ils ont pour motif un atome métalliques, les
        atomes n'ont pas de liaison directe, seulement un mise en commun
        d'électrons.
    \item[Cristaux ioniques] Ils sont constitués de cations et d'anions. Ils
        attirent les charges.
    \item[Cristaux covalents] Constitués d'atomes
\end{description}
Les modèles sont listés par odre croissant de force des liaisons. Ainsi, les
cristaux covalents requièrent plus d'énergie pour briser les liaisons.


\subsection*{Exemples de cristaux}
\begin{ex}[Cristaux métalliques]
    \begin{compactdesc}
        \item[CC] Fer $\alpha$, Chrome, Tungstène, \dots
        \item[CFC] Fer $\gamma$, Or, Ag, Aluminium, Cuivre, \dots
        \item[HC] Zinc, Titane, Cadmium, \dots
        \item[CP] Seulement le Polonium.
    \end{compactdesc}
\end{ex}
\begin{rem}
    Toutes les sphères font le même diamètre, donc le cubique simple n'est pas
    très stable et peu rencontré.
\end{rem}

\subsubsection*{Cristaux ioniques}
\begin{ex}[Chlorure de Césium CsCl cf figure~\ref{fig:3_cscl_struct} page~\pageref{fig:3_cscl_struct}]
    Un ion $\text{Cl}^-$ se trouve sur chaque coin du cube ($N_1 = 1$), un ion
    Césium $\text{Cs}^+$ se trouve au centre du cube ($N_2 = 1$).\\
    Cela donne une structure cubique centrée justifiée par les rayons des ions
    : $r(\text{Cl}^-) = 181\text{pm}, r(\text{Cs}^+) = 94 =\text{pm}$
\end{ex}
\begin{ex}[NaCl cf figure~\ref{sfig:3_nacl_cristal} page~\pageref{sfig:3_nacl_cristal}]
    Les ions Chlorure Cl$^-$ s'organisent en CFC ($N_1 = 4$), les ions Sodium
    Na$^+$ occupent le milieu de chaque arête ainsi que le centre du cube
    ($N_2 = 4$).\\
    Cela donne une structure cubique simple justifiée par les rayons des
    ions :
    $r(\text{Cl}^-) = 181\text{pm}, r(\text{Na}^+) = 97\text{pm}$
\end{ex}

\begin{rem}[Variétés allotropiques]
    Un corps peut exister sous différentes formes cristallines. Par exemple,
    le fer $\alpha$ et le fer $\gamma$ sont des variétés allotropiques du
    fer.
\end{rem}

\subsubsection*{Cristaux covalents}
\begin{ex}[Carbone cf~\ref{fig:carbone}]
    Sous forme de diamant (figure~\ref{fig:diamant}) chaque atome de
    carbone est lié à ses 4 voisins les plus proches et a donc une
    structure tétraédrique très robuste.\\
    Sous forme de graphite (figure~\ref{fig:graphite}) les atomes de
    carbone s'organisent en feuillets hexagonaux régulièrement
    espacés et décalés.
\end{ex}
\begin{figure}
    \centering
    \subfloat[Diamant]
        {\includegraphics[width=5cm]{diamant.png}\label{fig:diamant}}
    \qquad
    \subfloat[Graphite]
    {\includegraphics[width=6cm]{chap4/graphite.png}\label{fig:graphite}}
    \caption{Structures cristallines du carbone}\label{fig:carbone}
\end{figure}

\subsubsection*{Cristaux moléculaires}
\begin{ex}[Eau]
    Les molécules d'eau H$_2$0 s'agglomèrent suivant plusieurs structures,
    il y a plus d'une dizaine de variétés allotropiques. Un exemple
    est donné figure~\ref{fig:glace} page~\pageref{fig:glace}. Une molécule d'eau est placée
    à chaque noeud du réseau cristallin.
\end{ex}
