\documentclass[11pt,a4paper]{book}
\usepackage[a4paper,portrait]{typearea}
\usepackage[utf8]{inputenc}
\usepackage[T1]{fontenc}
\usepackage[francais]{babel}
\usepackage{amsfonts,amsmath,amssymb}
\usepackage{amsthm}
\usepackage{xfrac}
\usepackage{palatino}
\usepackage{ellipsis}
\usepackage[euler-digits]{eulervm}
\usepackage{fancybox,framed,ulem,array,multicol}
\usepackage[np]{numprint}
\usepackage{color,colortbl}
\usepackage{tabulary}
\usepackage{paralist}
\usepackage{tikz,xcolor,wrapfig,graphicx,subfig}
\usepackage{makeidx}
\usepackage{url}

\graphicspath{{pictures/}}
\DeclareUnicodeCharacter{00A0}{~}

\theoremstyle{plain}
\newtheorem{prop}{Propriété}


\theoremstyle{definition}
\newtheorem{defi}{Définition}


\theoremstyle{remark}
\newtheorem*{rem}{Remarque}
\newtheorem{ex}{Exemple}

\makeindex

\title{Cristallographie\\ \bsc{mpsi}}
\author{}
\date{}

\begin{document}
\maketitle
\frontmatter
\chapter{Préface}
Ce cours a été écrit à partir des documents et présentations fournis par Mme.\bsc{Brosse}
en Juin 2015 dans le cadre de l’enseignement de la cristallographie en MPSI . Sa création
a été motivée par la qualité des informations et des schémas contenus dans les
présentations qui n’ont pas été imprimées. Les documents sources n’ont pas été recopiés
tels quels et on fait l’objet d’un réarrangement et d’un enrichissement. Tout
commentaire permettant d’améliorer ce document pourra être envoyé par courrier électronique
à l’adresse \url{gabrielhondet@gmail.com} et sera apprécié.

\hfill Gabriel \bsc{Hondet}

\mainmatter
\tableofcontents
\listoffigures

\chapter{L'état solide cristallin}
\section{Les différents états solides}
\subsection{Cristal parfait}\index{cristal!parfait}
C'est le modèle idéal du solide cristallisé, il est constitué d'un arrangement
périodique d’entités (atomes, ions, ou molécules) dans les trois directions
de l’espace et à l’infini.
\begin{rem}
    Le cristal réel s'écarte de ce modèle\index{cristal!réel}
\end{rem}

\subsection{Cristal amorphe}\index{cristal!amorphe}
Un cristal est dit amorphe si l'ordre est local, et le désordre existe
à longue distance.
\begin{ex}[le verre]\index{verre}
    les verres sont le plus souvent obtenus par refroidissement d’un liquide
    n’ayant pas pu cristalliser; ils apparaissent comme un liquide figé 
    d’énorme viscosité. Cette forme de la matière est appelé état vitreux.
\end{ex}

\subsection{Cristal réel}\index{cristal!réel}
Le cristal réel présente souvent des défauts (absence d'une entité, déplacement
d'une entité par rapport à sa position parfaite \dots). Il n'est pas toujours 
constitué d’un unique cristal mais de plusieurs monocristaux juxtaposés
les uns aux autres, on l’appelle alors cristal polycristallin\index{polycristal}.
%(voir figure~\ref{fig:quartz_rutile}).
%\begin{figure}
%    \centering
%    \includegraphics[width=10cm]{Quartz_a_rutile.jpg}
%    \caption{Quartz à rutile}\label{fig:quartz_rutile}
%\end{figure}
\begin{rem}
    Ce cours ne développe que la structure du cristal parfait, c'est-à-dire un
    état \emph{cristallin parfaitement ordonné} à très grande distance
    (idéalement à l'infini).
\end{rem}


\section{Modèle du cristal parfait}\index{cristal!parfait}
\subsection{Motif}\index{motif}
Un cristal parfait résulte de la répétition périodique à l’infini, dans
les trois dimensions de l’espace, d’une entité élémentaire (atome, ion
ou molécule voire macromolécule) appelée \emph{motif}. Voir la
table~\ref{tab:1_motifs_ex} pour des exemples de motifs.
\begin{defi}
    Le motif d'un cristal est la plus petite entité (atome ou groupe d'atomes)
    qui se répète périodiquement.
\end{defi}
\begin{table}
    \centering
    \begin{tabular}{|l|c|c|c|c|}
        \hline Type de cristal & Métallique & Covalent & Moléculaire
        & Ionique \\
        \hline Motif & Atome & Atome & Molécule & Ions \\
        Exemple & Cu & C & H$_2$O & Na$^+$,Cl$^-$ \\ \hline
    \end{tabular}
    \caption{Exemples de motifs}\label{tab:1_motifs_ex}
\end{table}

\subsection{Réseau}\index{réseau}
La description d'un cristal requiert un support géométrique, c’est-à-dire un 
ensemble de points où on place le motif.
\begin{defi}
    Un réseau est un outil géométrique permettant de décrire l’arrangement
    périodique tridimensionnel de points appelés des nœuds.
\end{defi}
\index{n\oe ud}
\begin{figure}
    \centering
    \includegraphics[width=6cm]{reseau.png}
    \caption{Exemples de réseaux}
\end{figure}
Un cristal est l'association d'un réseau et d'un motif placé aux noeuds du
réseau.\index{cristal}

\subsection{Maille}\index{maille}
Les noeuds du réseau se déduisent les uns des autres par des opérations
de translation, combinaison linéaire de trois vecteurs non coplanaires
(usuellement notés $\vec{a}, \vec{b}, \vec{c}$). Ces
trois vecteurs définissent une structure parallélépipédique appelée
\emph{maille}.
\begin{defi}
    Une maille est une unité de base parallélépipédique à partir de laquelle
    on peut décrire tout le cristal en lui faisant subir différentes
    translations suivant les directions des trois axes cristallins.
\end{defi}

\subsection{Les sept systèmes cristallins}
Les réseaux cristallins peuvent être décrits à partir de \emph{7 mailles
élémentaires} qui définissent \emph{7 systèmes cristallins}.
\index{maille!élémentaire}
Selon que la maille est simple (un seul motif par maille) ou multiple, et à
partir de ces 7 systèmes cristallins, on définit
\emph{14 réseaux de \bsc{Bravais}} (cf table~\ref{tab:bravais}). On distingue
parmis les systèmes différents modes cristallins :
\index{Bravais, réseau de}\index{mode cristallin}
\begin{description}
    \item[P]  mode primitif, n\oe uds de réseaux seulement aux sommets de la 
        maille;\index{primitif|see{mode cristallin}}
    \item[I] un n\oe ud de réseau au centre du volume de la maille;
    \item[C] un n\oe ud de réseau au centre de la face C de la maille, définie par 
        les axes $\vec{a}$ et $\vec{b}$;
    \item[F] un n\oe ud de réseau au centre de chaque face de la maille.
\end{description}
\begin{rem}
    Il existe aussi les modes de réseaux A, B et R, mais ils ne seront pas étudiés
    dans ce cours (les modes A et B sont semblables au mode C).
\end{rem}
\begin{rem}\index{cubique à faces centrées|see{CFC}}\index{hexagonal compacte|see{HC}}
    Les structures cubiques F seront appelées \emph{CFC} pour \emph{cubique à
    faces centrées}. Les structures hexagonales P seront appelées \emph{HC}
    pour \emph{hexagonal compacte}.
\end{rem}

\begin{table}
    \footnotesize
    \begin{tabulary}{\linewidth}{|p{2.5cm}|c|c|c|c|}
        \hline \textbf{Système} & Type P & Type C & Type I & Type F \\\hline
        Cubique, \mbox{$a = b = c$}
        \mbox{$\alpha = \beta = \gamma = \sfrac{\pi}{2}$} &
        \includegraphics[width=2.2cm]{bravais_cp.png} & & 
        \includegraphics[width=2.2cm]{bravais_ci.png} &
        \includegraphics[width=2.2cm]{bravais_cf.png} 
        \\
        \hline Rhomboédrique $a = b = c$,
        \mbox{$\alpha = \beta = \gamma \neq \sfrac{\pi}{2}$}&
        \includegraphics[width=2.2cm]{bravais_rp.png} & & &
        \\
        \hline Hexagonal $a = b \neq c$
        \mbox{$\alpha = \beta = \sfrac{\pi}{2}$} $\gamma = \sfrac{2\pi}{3}$ &
        \includegraphics[width=2.2cm]{bravais_hp.png} & & &
        \\
        \hline Quadratique $a = b \neq c$
        \mbox{$\alpha = \beta = \gamma = \sfrac{\pi}{2}$} &
        \includegraphics[width=2.2cm]{bravais_qp.png} & &
        \includegraphics[width=2.2cm]{bravais_qi.png} &
        \\
        \hline Orthorombique $a \neq b \neq c$
        \mbox{$\alpha = \beta = \gamma = \sfrac{\pi}{2}$} &
        \includegraphics[width=2.2cm]{bravais_op.png} &
        \includegraphics[width=2.2cm]{bravais_oc.png} &
        \includegraphics[width=2.2cm]{bravais_oi.png} &
        \includegraphics[width=2.2cm]{bravais_of.png}
        \\
        \hline Monoclinique $a \neq b \neq c$
        \mbox{$\alpha=\gamma=\sfrac{\pi}{2} \neq \beta$}
        &
        \includegraphics[width=2.2cm]{bravais_mp.png} &
        \includegraphics[width=2.2cm]{bravais_mc.png} & &
        \\
        \hline Triclinique $a \neq b \neq c$
        \mbox{$\alpha \neq \beta \neq \gamma \neq \sfrac{\pi}{2}$} &
        \includegraphics[width=2.2cm]{bravais_tp.png} & & &
        \\
        \hline
    \end{tabulary}
    \caption{Les 14 réseaux de \bsc{Bravais}}\label{tab:bravais}
\end{table}

\subsection{Caractéristiques des réseaux cristallins}
\subsubsection{Volume d'une maille}\index{maille!volume}
\begin{prop}
    \begin{equation}
        V = [\vec{a}, \vec{b}, \vec{c}] = \det(\vec{a}, \vec{b}, \vec{c})
        = (\vec{a} \wedge \vec{b}) \cdot \vec{c}
        \label{eq:volume}
    \end{equation}
    Où $[\vec{a}, \vec{b}, \vec{c}]$ est le produit mixte.
\end{prop}

\subsubsection{Modèle des sphères dures}\index{sphères dures, modèle des}
Les atomes, ions ou molécules constituant le solide cristallin parfait seront
supposés être des sphères indéformables et compactes.

\subsubsection{Famille cristalline}
Ensemble des matériaux ayant même structure et présentant une forte analogie des
propriétés chimiques et physiques.

\subsubsection{Nombre de motifs par maille}\index{maille!motifs par maille}
Une maille est dite \emph{unitaire} si elle ne comporte qu'un seul motif,
\emph{multiple} si elle en comporte plusieurs.
\paragraph{Détermination du nombre de motifs par maille $Z$}
\begin{itemize}
    \item un élément extérieur à la maille compte pour $0$;
    \item un élément intérieur à la maille compte pour $1$;
    \item un élément au sommet de la maille compte pour $\frac{1}{8}$;
    \item un élément sur une arête compte pour $\frac{1}{4}$;
    \item un élément sur une face compte pour $\frac{1}{2}$.
\end{itemize}

\subsubsection{Coordinence}\index{coordinence}\index{voisins|see{coordinence}}
\begin{defi}[Coordinence de l'atome ou ion $A$ par rapport à $V$]
    nombre $x$ de plus proches voisins $V$ de $A$ noté $A/V = [x]$.
\end{defi}

\subsubsection{Masse volumique}\index{masse volumique}
\begin{defi}[Masse volumique ramenée à une maille]
    Rapport de la masse d'une maille et de son volume :
    \begin{equation}
        \rho = \frac{ZM}{\mathcal{N}_\text{A}V}\label{eq:masse_volumique}
    \end{equation}
    \begin{itemize}
        \item $Z$ nombre de motifs par maille;
        \item $M$ masse molaire du motif;
        \item $\mathcal{N}_\text{A}$ nombre d'Avogadro;
        \item $V$ volume de la maille.
    \end{itemize}
\end{defi}

\subsubsection{Compacité $C$}\index{compacité}
\begin{defi}
    Nombre sans dimension qui mesure le taux d'occupation réel de l'espace
    par les atomes ou ions assimilés à des sphères.\\
    Elle est toujours comprise entre 0 et 1.
    \begin{equation}
        C = \frac{4}{3}\frac{N \pi r^3}{V} \label{eq:comp}
    \end{equation}
    \begin{itemize}
        \item $N$ nombre de motifs;
        \item $r$ rayon des atomes ou ions;
        \item $V$ volume de la maille.
    \end{itemize}
\end{defi}

\subsection{Sites}\index{site}\index{cavité|see{site}}
Tout réseau cristallin de sphères identiques $S$ de rayon $R$ comporte
nécessairement des portions d'espace non occupées puisque $C < 1$; ils
constituent des sites. Le réseau constitué des sphères $S$ s'appelle le
\emph{réseau h\^ote}.\index{réseau!hôte}
\begin{defi}[Site intersticiel]
    Centre de polyèdre de coordination, régulier ou non, ayant pour sommets
    les centres des premiers voisins du réseau h\^ote
\end{defi}
\begin{ex}[cf figure~\ref{fig:2_cfc_sites} page~\pageref{fig:2_cfc_sites}]
    \begin{compactdesc}
        \item[Site cubique C] le polyèdre de coordination est un cube :
            $C/S = [8]$;\index{site!cubique}
        \item[Site octaédrique O] le polyèdre de coordination est un
            octaèdre à 6 sommets et 8 faces $O/S = [6]$;\index{site!octaédrique}
        \item[Site tétraédrique T] le polyèdre de coordination est un
            tétraèdre à 4 sommets et 4 faces $T/S = [4]$.\index{site!tétraédrique}
    \end{compactdesc}
\end{ex}


\section{Les différents modèles de cristaux}
\begin{description}
    \item[Cristaux moléculaires] Ce sont des cristaux ayant pour motif une
        molécule. Ils présentent des intéractions électriques;\index{cristal!moléculaire}
    \item[Cristaux métalliques] Ils ont pour motif un atome métalliques, les
        atomes n'ont pas de liaison directe, seulement une mise en commun
        d'électrons;\index{cristal!métallique}
    \item[Cristaux ioniques] Ils sont constitués de cations et d'anions. Ils
        attirent les charges;\index{cristal!ionique}
    \item[Cristaux covalents] Constitués d'atomes.\index{cristal!covalent}
\end{description}
Les modèles sont listés par odre croissant de force des liaisons. Ainsi, les
cristaux covalents sont ceux qui requièrent le plus d'énergie pour briser
les liaisons.\index{cohésion, énergie de}


\subsection*{Exemples de cristaux}
\begin{ex}[Cristaux métalliques]\index{cristal!métallique}
    \begin{compactdesc}
        \item[CC] Fer $\alpha$, Chrome, Tungstène, \dots
            \index{fer}
        \item[CFC] Fer $\gamma$, Or, Ag, Aluminium, Cuivre, \dots
            \index{cuivre}
        \item[HC] Zinc, Titane, Cadmium, \dots
        \item[CP] Seulement le Polonium.
    \end{compactdesc}
\end{ex}
\begin{rem}
    Toutes les sphères font le même diamètre, donc le cubique simple n'est pas
    très stable et peu rencontré.\index{cubique simple}
\end{rem}

\subsubsection*{Cristaux ioniques}\index{cristal!ionique}
\begin{ex}[Chlorure de Césium CsCl cf figure~\ref{fig:3_cscl_struct} page~\pageref{fig:3_cscl_struct}]
    \index{CsCl}
    Un ion $\text{Cl}^-$ se trouve sur chaque coin du cube ($N_1 = 1$), un ion
    Césium $\text{Cs}^+$ se trouve au centre du cube ($N_2 = 1$).\\
    Cela donne une structure cubique centrée justifiée par les rayons des ions
    : $r(\text{Cl}^-) = 181\text{pm}, r(\text{Cs}^+) = 94\text{pm}$.
\end{ex}
\begin{ex}[NaCl cf figure~\ref{fig:3_nacl_cristal} page~\pageref{fig:3_nacl_cristal}]
    \index{NaCl}
    Les ions Chlorure Cl$^-$ s'organisent en CFC ($N_1 = 4$), les ions Sodium
    Na$^+$ occupent le milieu de chaque arête ainsi que le centre du cube
    ($N_2 = 4$).
    Cela donne une structure cubique simple justifiée par les rayons des
    ions:
    $r(\text{Cl}^-) = 181\text{pm}, r(\text{Na}^+) = 97\text{pm}$.
\end{ex}

\begin{rem}[Variétés allotropiques]\index{allotropique, variété}
    Un corps peut exister sous différentes formes cristallines. Par exemple,
    le fer $\alpha$ et le fer $\gamma$ sont des variétés allotropiques du
    fer.
\end{rem}

\subsubsection*{Cristaux covalents}\index{cristal!covalent}
\begin{ex}[Carbone]\index{carbone}\index{diamant}\index{graphite}
    Sous forme de diamant (figure~\ref{fig:4_diamant_eclatee}
    page~\pageref{fig:4_diamant_eclatee}) chaque atome de
    carbone est lié à ses 4 voisins les plus proches et a donc une
    structure tétraédrique très robuste.
    Sous forme de graphite (figure~\ref{fig:4_graphite_3d}
    page~\pageref{fig:4_graphite_3d}) les atomes de
    carbone s'organisent en feuillets hexagonaux régulièrement
    espacés et décalés.\index{feuillet}
\end{ex}


\subsubsection*{Cristaux moléculaires}\index{cristal!moléculaire}
\begin{ex}[Eau]\index{eau|see{glace}}\index{H$_2$O|see{glace}}\index{glace}
    Les molécules d'eau H$_2$O s'agglomèrent suivant plusieurs structures,
    il y a plus d'une dizaine de variétés allotropiques.
    \index{allotropique, variété}Un exemple
    est donné figure~\ref{fig:4_glace} page~\pageref{fig:4_glace}. Une molécule d'eau est placée
    à chaque noeud du réseau cristallin.
\end{ex}



\chapter{Les cristaux métalliques}\index{cristal!métallique|(}
L'expérience montre que les cristaux métalliques sont parmis les plus
compactes. Leur structure spatiale sera décrite en faisant l'hypothèse
que les motifs sont équivalents à des sphères indéformables, de rayon
$r$ et l'empilement des motifs est le plus compact possible (les
sphères sont au contact).\index{compacité}\index{sphères dures, modèle des}

\section{Les structures cristallines compactes}
\index{coordinence!structure compacte}
Toutes les structures cristallines compactes ont une coordinence de
12, voir figure~\ref{fig:coordinence_cfc}.
\subsection{Structure CFC}\index{CFC}
Trois plans A, B, C (cf figure~\ref{fig:plans_métal}) se répètent
dans l'ordre A,B,C,A,B, \dots constituant une structure cubique à face
centrée, voir figure~\ref{fig:maille_métal_couches}.
La figure~\ref{fig:maille_cfc} offre une vue simplifiée de la maille.
\begin{figure}
    \centering
    \subfloat[Plan A]
    {\includegraphics[width=4cm]{chap2/plan_a.png}}
    \qquad
    \subfloat[Plan B]
    {\includegraphics[width=4cm]{chap2/plan_b.png}}
    \qquad
    \subfloat[Plans A,B et C]
    {\includegraphics[width=3cm]{chap2/plan_c.png}}
    \caption{Plans de la structure CFC}\label{fig:plans_métal}
\end{figure}
\begin{figure}
    \centering
    \subfloat[Vue en perspective]
    {\includegraphics[width=6cm]{chap2/crist_met.png}}
    \qquad
    \subfloat[Vue plane]
    {\includegraphics[width=6cm]{chap2/plans_met.png}}
    \caption{Maille cfc
    en fonction des couches}\label{fig:maille_métal_couches}
\end{figure}
\begin{figure}
    \centering
    \input{pictures/cfc.pdf_tex}
    \caption{Maille d'une structure cfc}\label{fig:maille_cfc}
\end{figure}

\subsection{Structure HC}\index{HC}
Deux plans A, B (cf figure~\ref{fig:plan_ab_hc}) se répètent A,B,A, \dots
formant une structure hexagonale compacte.
\begin{figure}
    \centering
    \subfloat[Vue éclatée]
    {\includegraphics[width=4cm]{chap2/plans_ab_hc_eclatee.png}
    \label{sfig:2_plans_ab_hc_éclatée}}
    \qquad
    \subfloat[Vue compacte]
    {\includegraphics[width=4cm]{chap2/plans_ab_hc_compacte.png}
    \label{sfig:2_plans_ab_hc_compacte}}
    \caption{Plans de la structure HC}\label{fig:plan_ab_hc}
\end{figure}
En se ramenant aux réseaux de Bravais (tableau\ref{tab:bravais}),
la maille est un volume engendré par la translation d'un losange,
visible sur la figure~\ref{fig:maille_hc}.
\begin{figure}
    \centering
    \subfloat[Base de la maille]
    {\includegraphics[width=4cm]
        {chap2/maille_hc_2d.png}\label{sfig:maille_hc_2d}}
    \qquad
    \subfloat[Vue en perspective de la maille]
    {\includegraphics[width=3cm]
        {chap2/maille_hc_3d.png}\label{sfig:maille_hc_3d}}
    \caption{Vues d'une maille de structure hc}\label{fig:maille_hc}
\end{figure}

\begin{figure}
    \centering
    \subfloat[Structure CFC]
    {\includegraphics[width=4cm]{chap2/plans_cfc.png}
    \label{sfig:2_plans_cfc_couleur}}
    \qquad
    \subfloat[Structure HC]
    {\includegraphics[width=4cm]{chap2/plans_hc.png}
    \label{sfig:2_plans_hc_couleur}}
    \caption{Structures CFC et HC}
\end{figure}
\begin{figure}
    \centering
    \includegraphics[width=9cm]{chap2/cristaux_metalliques.png}
    \caption{Structures métalliques}
\end{figure}


\section{Structure CFC}\index{CFC|(}
Les paramètres suivants seront utilisés :
\begin{description}
    \item[$a$] taille caractéristique d'une maille
    \item[$r$] rayon du motif (sphère représentant l'atome ou ion)
    \item[$Z$] nombre de motifs par maille
    \item[$C$] compacité
    \item[$\lbrack x \rbrack$] coordinence
    \item[$V$] volume d'une maille
    \item[$\rho$] masse volumique
\end{description}
\begin{figure}
    \centering
    \includegraphics[width=4cm]{chap2/coordinence_cfc.png}
    \caption{Plus proches voisins d'un atome dans la
        structure cfc}\label{fig:coordinence_cfc}
\end{figure}
\subsection{Paramètre géométrique}
\begin{figure}
    \centering
    \includegraphics[width=4cm]{chap2/prm_geo_cfc.png}
    \caption{Représentation d'une maille en fonction de $a$}\label{fig:prm_geo_cfc}
\end{figure}

La diagonale d'une maille mesure $\sqrt{2}a$ (par Pythagore).
Les sphères étant tangentes le long de la diagonale, on obtient :
\begin{gather}
    4r = \sqrt{2}a \\
    \boxed{r = \frac{a}{2^{\sfrac{3}{2}}}} \label{eq:r}
\end{gather}
\subsection{Nombre de motifs par maille}
Voir figure~\ref{fig:maille_cfc}.
\begin{itemize}
    \item Les 8 motifs aux 8 sommets sont chacun partagés
        entre 8 mailles
    \item Les 6 motifs sur les 6 faces sont chacun partagés
        entre 2 mailles
\end{itemize}
\begin{equation}
    Z = 8 \times \frac{1}{8} + 6 \times \frac{1}{2} = 4
\end{equation}
\subsection{Compacité}
La formule de la compacité~\ref{eq:comp} nous donne
\begin{equation*}
    C = \frac{4 \frac{4}{3} \pi r^3}{a^3}
\end{equation*}
d'où, en remplaçant $r$ avec~\ref{eq:r},
\begin{equation}
    \boxed{C = \frac{\pi}{3\sqrt{2}} \approx 0.74}
\end{equation}

\subsection{Sites}
Compte tenu de la compacité, il existe des sites entre les motifs.
\index{compacité}
\begin{rem}
    Ne seront étudiés que les sites de symétrie importante,
    à savoir les sites octaédriques et tétraédriques.
\end{rem}
\begin{figure}
    \centering
    \input{pictures/ex_sites.pdf_tex}
    \caption[Sites d'une structure cfc]
    {Site octaédrique (rouge) et tétraédrique (bleu)}\label{fig:2_cfc_sites}
\end{figure}
\subsubsection{Sites octaédriques}\index{site!octaédrique}
Les sites octaédriques sont au centre de la maille et au milieu
des arêtes. On a donc $1 + 12 \times \frac{1}{4} = 4$ sites octaédriques.
\begin{figure}
    \centering
    \includegraphics[width=3.5cm]{chap2/sites_o_cfc.png}
    \caption{Placement d'une entité dans un site octaédrique}\label{fig:site_o_cfc}
\end{figure}
On peut calculer la taille maximale des atomes étrangers.
Soit $r_{\text{O,max}}$ le rayon maximal d'un atome étranger.
Les sphères (atomes h\^otes et étrangers) étant tangentes le
long du c\^oté de la maille, on a :\index{réseau!hôte}
\index{étranger, atome}
\begin{equation}
    a = 2r + 2r_{\text{O,max}}
\end{equation}
d'où, en remplaçant $a$ à l'aide de~\ref{eq:r} :
\begin{equation}
    \boxed{r_{\text{O,max}} = (\sqrt{2} - 1)r} \label{eq:rmax}
\end{equation}
\begin{rem}
    Il sera convénient d'utiliser le rapport noté $x$ :
    \begin{equation*}
        \frac{r_{\text{O,max}}}{r} = x
    \end{equation*}
    Dans le cas d'une CFC :
    \begin{equation*}
        x = \sqrt{2} - 1
    \end{equation*}
\end{rem}

\begin{ex}[Cas du fer $\gamma$]\index{fer}
    Données : $r = 124\text{pm}, M = 56\text{g.mol}^{-1}$
    \begin{compactitem}
        \item Déterminer le paramètre de maille $a$;
        \item Déterminer le rayon maximal des atomes pouvant s'insérer dans un
            site octaédrique $r_{\text{O,max}}$;
        \item Peuvent des atomes d'hydrogène s'insérer ($r_{\text{H}} = 37$pm)?
            \index{hydrogène}
        \item Déterminer la masse volumique $\rho$.
    \end{compactitem}
    L'équation~\ref{eq:r} donne
    $a=2^{\sfrac{3}{2}} \times 124\text{pm} = 351\text{pm}$\\
    Pour trouver $r_{0,\text{max}}$ on utilise~\ref{eq:rmax} :
    $r_{\text{O,max}} = (\sqrt{2} - 1)\times 124 = 51\text{pm}$.\\
    On observe que $r_{\text{H}} \leq r_{0,\text{max}}$, des atomes
    d'hydrogènes peuvent donc se positionner dans les sites
    octaédriques. En revanche, l'insertion d'atomes d'azote ou de
    carbone (rayons $70$ et $77$pm) entraîne une déformation importante
    du métal. Le fer nitrué ou carburé (la fonte) sont nettement
    plus fragiles que le fer pur.\\
    \index{propriétés mécaniques!fer}\index{fer}
    D'après~\ref{eq:masse_volumique} :
    $\rho = \frac{4 \times 56}{\np{6,02}.10^{23}(351.10^{-12}.10^2)^3} = \np{8,60}\text{g.cm}^{-1}$
\end{ex}

\subsubsection{Sites tétraédriques}\index{site!tétraédrique}
Les centres des sites tétraédriques sont situés aux centres des
huit petits cubes en lesquels on peut découper le cube d’arête
$\frac{a}{2}$.
\paragraph{Calcul de $r_{\text{T,max}}$}
Les sphères sont tangentes le long de la diagonale du cube.
En notant $D$ la diagonale du cube, on a $D = \sqrt{3}a$
(par Pythagore). La diagonale $D'$ d'un cube d'arête $\frac{a}{2}$
vaut donc $D' = D\div 2 = \frac{\sqrt{3}a}{2}$
Comme le site est au milieu du cube d'arête $\frac{a}{2}$, on a :
\begin{equation*}
    2r + 2r_{\text{T,max}} = D'
\end{equation*}
c'est-à-dire, en remplaçant :
\begin{equation}
    \boxed{r_{\text{T,max}} = r(\sqrt{\frac{3}{2}} - 1)}\label{eq:rmax_tetra}
\end{equation}
D'où,
\begin{equation}
    x = \sqrt{\frac{3}{2}} - 1
\end{equation}
\index{CFC|)}


\section{Structure HC}\index{HC|(}
Les métaux alcalino-terreux et de nombreux métaux de transition
adoptent la structure hexagonale compacte.
\index{métal!alcalino-terreu}\index{métal!transition, de}
\begin{description}
    \item[$a$] taille caractéristique d'une maille
    \item[$c$] hauteur de la maille
    \item[$r$] rayon du motif (sphère représentant l'atome ou ion)
    \item[$Z$] nombre de motifs par maille
    \item[$C$] compacité
    \item[$\lbrack x \rbrack$] coordinence
    \item[$V$] volume d'une maille
    \item[$\rho$] masse volumique
\end{description}
\subsection{Paramètre géométrique}
Vue la figure~\ref{sfig:maille_hc_2d}, il est évident que :
\begin{equation}
    a = 2r \label{eq;a_r_hc}
\end{equation}
La maille n'étant pas cubique, il reste le paramètre géométrique
de hauteur $c$ visible sur la figure~\ref{sfig:maille_hc_3d}.
On se place dans le tétraèdre dont la base est le triangle formé
par 3 motifs de la base de la maille et le sommet est le motif
de la deuxième couche. Ce tétraèdre sera paramétré selon le
schéma~\ref{fig:prm_geo_hc}.\\
On notera $h$ pour la hauteur, donc $h = SH = \frac{c}{2}$.\\
Par Pythagore, on a :
\begin{equation}
    a^2 = \text{PH}^2 + h^2 \label{eq:PH_a_h}
\end{equation}
Comme H est le centre de gravité du triangle PQR, on a
\begin{equation}
    \text{PH} = \frac{2}{3}\text{PN}\label{eq:PG_PN}
\end{equation}
or, en considérant l'angle formé par (PN) et (RN) :
\begin{equation*}
    \cos\Bigl( \frac{\pi}{6}\Bigr) = \frac{\text{PN}}{a}
\end{equation*}
D'où, en remplaçant PN à l'aide de~\ref{eq:PG_PN}:
\begin{equation}
    \text{PH} = \frac{\sqrt{3}a}{3}\label{eq:PH_a}
\end{equation}
En remplaçant PH dans~\ref{eq:PH_a_h}, on obtient :
\begin{equation*}
    h = \sqrt{\frac{2}{3}}a
\end{equation*}
Soit finalement
\begin{equation}
    \boxed{c=2\sqrt{\frac{2}{3}}a} \label{eq:c_hc}
\end{equation}
\begin{figure}
    \centering
    \includegraphics[width=5cm]{chap2/prm_geo_hc.png}
    \caption{Tétraèdre dans une maille
        de structure HC}\label{fig:prm_geo_hc}
\end{figure}
\begin{rem}
    Un cristal ayant une structure hexagonale compacte
    peut être caractérisé par le rapport $\frac{c}{a}$.
    Ici, $\frac{c}{a} = 2\sqrt{\frac{2}{3}}\approx \np{1.63}$.
    Pour le magnésium, le rapport vaut
    $\np{1,62}$.Cette valeur est très proche de celle
    obtenue ci-dessus, les atomes sont des sphères
    au contact. Pour le titane,
    $\frac{c}{a}=\np{1,59}$ et pour le cadmium,
    $\frac{c}{a}=\np{1,89}$. Les écarts sont donc
    importants. Dans le cadre d’une structure par empilement,
    les motifs ne sont plus des sphères mais des ellipsoïdes
    aplaties dans le cas du cadmium, étirées dans le cas du titane.
    \index{sphères dures, modèle des}
\end{rem}

\subsection{Nombre de motifs par maille}
\begin{itemize}
    \item Les 8 motifs aux 8 sommets sont chacun partagés entre
        8 mailles
    \item Un motif entièrement dans la maille situé à
        $\sfrac{c}{2}$ et aligné avec le centre de gravité d'un
        des deux triangles formés par 3 des motifs de la base de
        la maille
\end{itemize}
\begin{equation}
    Z = \frac{1}{8} \times 8 + 1 = 2
\end{equation}

\subsection{Compacité}
La formule~\ref{eq:comp} donne:
\begin{equation*}
    C = \frac{2 \times \frac{4}{3}\pi r^3}{V}
\end{equation*}
Or d'après~\ref{eq:volume} :
\begin{equation*}
    V=a^2 \frac{\sqrt{3}}{2}
\end{equation*}
Donc en remplaçant $a$ avec~\ref{eq;a_r_hc} :
\begin{equation}
    V=2^3\sqrt{2}r^3 \label{eq:V_hc}
\end{equation}
D'où :
\begin{equation}
    \boxed{C = \frac{\pi}{3\sqrt{2}} \approx \np{0,74}}
    \label{eq:comp_hc}
\end{equation}

\subsection{Sites}
\begin{rem}
    L'étude détaillée des sites de la structure hexagonale
    compacte n'est pas au programme officiel.
\end{rem}
Un cristal ayant une structure hc possède des cavités comparables
à celles mises en évidence dans la structure cfc.
\begin{itemize}
    \item 2 cavités octaédriques
    \item 4 cavités tétraédriques
\end{itemize}
Compte tenu de l’analogie structurale qui existe entre les deux
systèmes, on admet que la taille des cavités de la structure
hc prend la même valeur que pour les cavités correspondantes
de la structure cfc. Il y a autant de cavités octaédriques que
de motifs dans la maille, soit deux.
\begin{figure}
    \centering
    \input{pictures/sites_hc.pdf_tex}
    \caption{Sites tétraédriques et octaédriques
        de la structure hc}
\end{figure}\index{HC|)}


\section{Exemples}
\begin{ex}[Structure cfc : le fer $\alpha$]\index{fer}\index{CFC}
    $a = 356$pm,
    $M = \np{55,8}$g.mol$^{-1}$.
    \begin{compactitem}
        \item Exprimer $r$ en fonction de $a$ paramètre
            de la maille;
        \item Déterminer le nombre de motifs par maille;
        \item Déterminer la compacité.
    \end{compactitem}
    D'après~\ref{eq:r} : $r = a\frac{\sqrt{2}}{4} \approx \np{124,9}$pm\\
    $Z = 4$\\
    D'après~\ref{eq:comp_hc}, $C = \frac{\pi}{3\sqrt{2}} \approx \np{0,74}$
\end{ex}
\begin{ex}[Structure hc : le magnésium]\index{HC}
    $a = 320$pm,
    $M = \np{24,3}$g.mol$^{-1}$
    \begin{compactitem}
        \item Déterminer la hauteur $c$ de la maille hexagonale
            compacte;
        \item Déterminer la relation entre $a$ et $r$ le rayon
            d'un atome du métal;
        \item Déterminer la coordinence et le nombre de motifs
            maille;
        \item Déterminer la compacité.
    \end{compactitem}
    D'après~\ref{eq:c_hc} : $c = \np{552,6}$pm,\\
    D'après~\ref{eq;a_r_hc} $a = 2r$,\\
    $Z = 2$, $[x] = 12$
\end{ex}



\section{Structure CC}\index{CC|(}
Les métaux alcalins et alcalino-terreux, ainsi que quelques
\index{métal!alcalin}\index{métal!alcalino-terreu}\index{fer}
\index{métal!transition, de}
métaux de transition comme le fer $\alpha$ cristallisent selon la structure cc.
En accord, avec le fait que ces structures sont moins compactes,\index{compacité}
il est à signaler que les métaux correspondants (comme le sodium) sont\index{sodium}
plus fragiles et plus malléables.

\subsection{Construction par empilement}
\begin{description}
    \item[Plan A] pavage en plaçant les sphères aux sommets de carrés
        de côté $a$ avec $a>2r$: les sphères ne sont pas au contact.
        Voir figure~\ref{sfig:plana_cc}.
    \item[Plan B] les sphères se placent dans les creux laissés par 
        les sphères du plan A, leurs centres se projettent sur
        les centres des carrés du plan initial.
        Voir~\ref{sfig:planb_cc_dessus} et~\ref{sfig:planb_cc_cote}.
    \item[Plan C] voir plan A et figure~\ref{sfig:planc_cc}
\end{description}
\begin{figure}
    \subfloat[Plan A]
    {\includegraphics[width=3cm]
        {chap2/plana_cc.png}\label{sfig:plana_cc}}
    \qquad
    \subfloat[Plan B, vue de dessus]
    {\includegraphics[width=3cm]
        {chap2/planb_cc_1.png}\label{sfig:planb_cc_dessus}}
    \qquad
    \subfloat[Plan B, vue de c\^oté]
    {\includegraphics[width=3cm]
        {chap2/planb_cc_2.png}\label{sfig:planb_cc_cote}}
    \qquad
    \subfloat[2\ieme plan A]
    {\includegraphics[width=3cm]
        {chap2/planc_cc.png}\label{sfig:planc_cc}}
    \caption{Plans de la structure CC}\label{plans_cc}
\end{figure}

\subsection{Coordinence}
Voir la figure~\ref{fig:coordinence_cc}.\\
$[x] = 8$
\begin{figure}
    \centering
    \includegraphics[width=3cm]{chap2/coordinence_cc.png}
    \caption{Maille CC}\label{fig:coordinence_cc}
\end{figure}

\subsection{Nombre de motifs par maille}
Voir figure~\ref{fig:coordinence_cc}.
\begin{itemize}
    \item Les 8 motifs aux 8 sommets sont partagés entre 8 mailles
    \item Le motif au milieu du cube n'est pas partagé
\end{itemize}
\begin{equation}
    Z = \frac{1}{8} \times 8 + 1 = 2
\end{equation}

\subsection{Compacité}
\begin{equation*}
    C = \frac{2\times \frac{4}{3} \pi r^3}{a^3}
\end{equation*}
D'où,
\begin{equation}
    C = \pi \frac{\sqrt{3}}{8} \approx \np{0,68}
\end{equation}
\begin{rem}
    La compacité de la structure cc est inférieure à celle
    des structures hc et cfc
\end{rem}
\begin{ex}[Na]\index{Na|see{sodium}}\index{sodium}
    $r = 190$pm, $M = \np{23,0}$g.mol$^{-1}$
\end{ex}

\section{Propriétés physiques}
Un élément possédant une faible électronégativité peut former
\index{électronégativité}
à l’état solide un cristal métallique dont la cohésion est
assurée par la liaison métallique.
Dans un modèle simple, on peut considérer que les électrons
de valence des atomes sont délocalisés sur l’ensemble du
\index{valence}
cristal, interagissant avec les cations métalliques situés
aux n\oe uds du réseau (modèle du gaz d’électrons)\index{gaz d'électrons}.
Les cations sont assimilés à des sphères dures
indéformables. La structure est imposée par les ions positifs.
La liaison métallique est forte (énergie de 100 à 800 kJ.mol$^{-1}$)
\index{cohésion, énergie de!CC}
et non directionnelle. Elle est à l’origine de certaines
\index{directionnelle, liaison}
propriétés macroscopiques des métaux :
\begin{itemize}
    \item température de fusion élevée \index{fusion, température de!CC}
        \index{CC!fusion, température de}
    \item conductivités thermique et électrique élevées
        \index{conductivité électrique}
    \item propriétés mécaniques de dureté\footnote{résistance à
        la pénétration}, de malléabilité\footnote{aptitude à la
        déformation sans se rompre} et de
        ductilité\footnote{aptitude au laminage et au filage}
        \index{propriétés mécaniques!métaux}
        \index{solidité|see{propriétés mécaniques!métaux}}
        \index{dureté|see{propriétés mécaniques!métaux}}
        \index{malléabilité|see{propriétés mécaniques!métaux}}
        \index{ductilité|see{propriétés mécaniques!métaux}}
    \item propriétés optiques d'opacité et de pouvoir réflecteur
        \index{propriétés optiques}
\end{itemize}

\subsection*{Alliages}\index{alliage}
Les cristaux métalliques peuvent conduire à plusieurs types
d'alliage :
\begin{itemize}
    \item lorsque les sites intersticiels de la structure sont
        occupés par des atomes de taille adéquate
    \item lorsque certains atomes de la structure sont remplacés
        par des atomes métalliques de nature différente
\end{itemize}
\begin{ex}[Alliages de fer]\index{fer!alliages}
    \begin{compactdesc}
        \item[Fonte] fer et carbone, carbone à plus de $\np{2,1}$\%
            et jusqu'à $\np{6,7}$\% en masse
        \item[Acier] fer et carbone (moins de $\np{2,1}$\% en masse
            de carbone) et traces éventuelles de nickel, chrome,
            molybdène en faible pourcentage ($<4$\%)
    \end{compactdesc}
\end{ex}
\begin{ex}[Alliages de cuivre]\index{cuivre!alliages}
    \begin{compactdesc}
        \item[Bronze] cuivre et étain (l'airin est l'ancien nom
            du bronze)
        \item[Laiton] cuivre et zinc
    \end{compactdesc}
\end{ex}
\index{CC|)}
\index{cristal!métallique|)}




\chapter{Les cristaux ioniques}
\section{Modèle de description}

\chapter{Cristaux covalents et moléculaires}\index{cristal!covalent|(}
Le programme officiel ne demande que de <<relier les caractéristiques des liaisons
covalentes, des interactions de Van der Waals et des liaisons hydrogènes et les
\index{Van der Waals}
propriétés macroscopiques des solides correspondants>>. Quatre exemples seront étudiés
brièvement dans le cadre du programme.

\section{Exemple de structure covalente : le diamant \index{diamant|(}}
\subsection{Description de la maille}
Les atomes de carbone constituent une structure cubique à faces centrées avec
occupation d’un site tétraédrique sur deux par un atome de carbone (cf blende). Chaque
\index{blende}\index{carbone}
atome de carbone a un environnement tétraédrique de quatre carbones : la distance
entre deux carbones voisins est $d_\text{C-C} = 154$pm.

\begin{figure}
    \centering
    \includegraphics[width=5cm]{chap4/diamant.png}
    \caption{Vue éclatée d'une maille de diamant}\label{fig:4_diamant_eclatee}
\end{figure}

\subsection{Caractéristiques de la maille}
\subsubsection{Nombre de motifs par maille}
Strucuture cfc : motifs aux sommets et au centre des faces, avec occupation de la
moitié des 8 sites tétraédriques.
\begin{equation}
    Z = 8 \times \frac{1}{8} + 6 \times \frac{1}{2} + 4 \times 1 = 8
\end{equation}

\subsubsection{Coordinence}
La coordinence C/C est de 4.

\subsubsection{Paramètre géométrique}
Les atomes de carbones les plus proches sont selon la diagonale d’un cube d’arête
$\frac{a}{2}$ on obtient :
\begin{equation}
    \frac{a\sqrt{3}}{4} = d_\text{C-C}
\end{equation}
D'où $a = 356$pm.

\subsubsection{Compacité}
\begin{equation}
    C = \frac{8\frac{4}{3}\pi r^3}{a^3} = \np{0,34}
\end{equation}
C'est une structure peu compacte.


\subsection{Propriétés physiques}
Le carbone (Z = 6) possède 4 électrons de valence sur sa couche externe. Pour\index{valence}
satisfaire à la règle de l’octet, il est tétravalent. La tétravalence est réspectée dans la structure
\index{octet, règle de}
du diamant : chaque atome de carbone échange avec ses quatre voisins quatre liaisons
de covalence. Ces liaisons assurent une grande stabilité au diamant : il en résulte que
le diamant est isolant, rigide et dur par suite de la forte intéraction entre les atomes de
carbone.\index{conductivité électrique!diamant}Son énergie \index{cohésion, énergie de!diamant}
de cohésion est de 717kJ/mol, sa température\index{fusion, température de!diamant} 
de fusion est 3500\degre C.\index{propriétés mécaniques!diamant}
\index{diamant|)}


\section{Autre structure covalente: le graphite}\index{graphite|(}
\subsection{Description de la maille}
Le graphite donne une structure en feuillets. Les atomes de carbone de type trigonal
\index{feuillet}\index{HC}
forment une structure hexagonale plane et régulière. Dans ce plan, les atomes voisins
sont équidistants (142pm). Les plans sont parallèles, distants de 335pm. Le passage
d’un plan A à un plan B se fait par translation de 141pm du plan A suivant un côté de
l’hexagone. Une translation de $-$141pm appliquée à B permet de retrouver le plan A.
Les plans se succèdent AB, AB, \dots
\begin{figure}
    \centering
    \includegraphics[width=8cm]{chap4/graphite_3d.png}
    \caption{Plans du graphite}\label{fig:4_graphite_3d}
\end{figure}

\subsection{Caractéristiques de la maille}
\subsubsection{Nombre de motifs par maille}
\begin{itemize}
    \item 8 sommets communs à 8 mailles;
    \item 2 atomes situés sur les faces;
    \item 4 atomes sur les arêtes;
    \item 1 atome à l'intérieur.
\end{itemize}
\begin{equation}
    Z = 8 \times \frac{1}{8} + 2 \times \frac{1}{2} + 4 \times
    \frac{1}{4} + 1 \times 1 = 4
\end{equation}
\subsubsection{Coordinence}
Chaque atome de carbone est trigonal, la coordinence
est donc de 3.
\subsubsection{Paramètre géométrique}
La distance entre deux atomes de carbones est $d_\text{C-C} = 142$pm.
Le rayon atomique vaut $r = 71$pm.

\subsection{Propriétés physiques}
Dans un plan donnée, chaque atome de carbone ne forme que trois
liaisons de covalence avec trois atomes de carbone voisins.
Les différents feuillets interagissent par des forces de Van der
Waals faisant intervenir l'interaction des électrons délocalisés
\index{Van der Waals}\index{feuillet}
d'un feuillet avec les noyaux des atomes de carbone des feuillets
les plus proches.
Il en résulte que le graphite est conducteur et mou (il se fend
facilement le long de ses feuillets).\index{conductivité électrique!graphite}
\index{propriétés mécaniques!graphite}\index{graphite!propriétés mécaniques}
Cette structure en feuillets permet aux plans de glisser les uns
par rapport aux autres ce qui explique les propriétés lubrificatrices
du graphite (huiles graphitiques). C'est aussi un cristal anisotrope,
la conduction électrique est 200 fois plus élevée dans une direction
parallèle à un feuillet que dans la direction orthogonale.\index{anisotropie}
\index{graphite|)}
\index{cristal!covalent|)}


\section{Structure moléculaire : la glace}\index{glace|(}\index{cristal!moléculaire|(}
\subsection*{Cristal moléculaire}
Un cristal moléculaire résulte de la juxtaposition de molécules
qui gardent leur identité dans le cristal (comme à l'état gazeux).
De ce fait, les liaisons covalentes interatomiques au sein des
molécules restent pratiquement inchangées par rapport à l'état
gazeux.\index{gaz}
On distingue plusieurs types de cristaux moléculaires selon la
nature de l'interaction qui en assure la cohésion :
forces de Van der Waals ou liaison à hydrogène.
\index{Van der Waals}

\begin{rem}[Liaisons hydrogène]\index{liaison hydrogène}
    De 10 à 30kJ/mol : force intermoléculaire impliquant un
    \index{cohésion, énergie de!liaison hydrogène}
    atome d'hydrogène et un atome électronégatif.
    \index{électronégativité}
\end{rem}
\begin{figure}
    \centering
    \includegraphics[width=4cm]{chap4/liaison_h.png}
    \caption[Liaison hydrogène]{Représentation d'une liaison
        hydrogène (en bleu)}\label{fig:4_liason_h}
\end{figure}
\begin{rem}[Forces de Van der Waals]\index{Van der Waals}
    Nommées en l'honneur du physicien néerlandais Johannes
    Diderik van der Waals (1837\---1923) prix nobel de
    physique en 1910. Il fut le premier à introduire les
    effets de ces forces dans les équations d'état des gaz
    en 1873
\end{rem}
\begin{figure}
    \centering
    \includegraphics[width=9cm]{glace.png}
    \caption{Structure cristalline de l'eau}\label{fig:4_glace}
\end{figure}

\subsection{Description de la structure}
La structure de la glace est semblable à celle du diamant.
\index{diamant}
La glace se présente selon les conditions de température et
de pression sous plusieurs variétés allotropiques.
\index{allotropique, variété}
La variété stable dans les conditions ordinaires est
de type hexagonal. La variété de glace cubique s’observe
sous de très faibles pressions dans l’intervalle de
température 148\--188K.


Dans la glace de type hexagonal, les atomes d’oxygène
ont une structure diamant: occupation des nœuds d’un
réseau cfc, avec occupation en alternance d’un\index{CFC}
site tétrédrique sur deux.\index{hydrogène}\index{oxygène}
Les atomes d’hydrogène sont situés sur les segments
joignant deux atomes d’oxygène plus proches voisins,
mais à distance $d_1=98$pm d’un atome d’oxygène -- selon
une liaison covalente -- et à distance $d_2= 177$pm
de l’autre atome selon une liaison hydrogène.\index{liaison hydrogène}
Voir figure~\ref{fig:4_tétraèdre_glace} et
figure~\ref{fig:4_glace} où les liaisons hydrogène sont
représentées par les traits entre un atome d'hydrogène
et un atome d'oxygène.
\begin{figure}
    \centering
    \includegraphics[width=6cm]{chap4/glace.png}
    \caption{Détail de la maille de glace}
    \label{fig:4_tétraèdre_glace}
\end{figure}

\subsection{Cohésion}
Dans la glace, les liaisons dirigées résultent de liaisons
\index{directionnelle, liaison}
hydrogène. Il en résulte une structure lacunaire qui permet
la dissolution de composés gazeux et la formation
\index{solubilité}d’<<hydrates de gaz>>.

\subsection{Paramètre géométrique}
On notera :
\begin{description}
    \item[$a$] arête du cube de la maille;
    \item[$d_\text{O-O}$] la distance entre deux atomes
        d'oxygène;
    \item[$d_1$] distance entre un atome d'hydrogène et
        l'oxygène le plus proche;
    \item[$d_2$] distance entre un atome d'hydrogène
        et un atome d'oxygène liés par une liaison hydrogène;
    \item[$\rho$] la masse volumique;
    \item[$\mathcal{N}_A$] le nombre d'Avogadro;
    \item[$M_{\text{H}_2\text{O}}$] la masse molaire de
        l'eau.
\end{description}
On a :
\begin{equation}
    d_\text{O-O} = \frac{a\sqrt{3}}{4} = d_1 + d_2
    = 275\text{pm}
\end{equation}

\subsection{Masse volumique}
\begin{equation}
    \rho = \frac{8M_{\text{H}_2\text{O}}}{\mathcal{N}_A
        a^3} = \np{9,33}.10^2\text{kg.m}^{-3}
\end{equation}
\begin{rem}
    L'eau est une des rares substances pour laquelle la
    masse volumique du solide est inférieure à celle du
    liquide.\index{masse volumique}
\end{rem}
\index{glace|)}


\section{Cristaux moléculaires de Van der Waals}\index{Van der Waals}
\index{cristal!Van der Waals, de}
Les gaz nobles, constitués de molécules monoatomiques
\index{gaz noble}
cristallisent, sauf l’hélium, dans la structure cfc.
Le diiode, le dibrome, le dioxyde de carbone, ont
une structure dérivée de la structure cfc du\index{CFC}
fait de molécules non sphériques.\index{sphères dures, modèle des}
\subsection{Cohésion}
La cohésion est assurée par des interactions de
Van der Waals. Celles ci sont de faible énergie:
10 à 100 fois moindre que celle des liaisons covalentes.
\index{cohésion, énergie de!cristal de Van der Waals}
Elles sont son dirigées, par conséquent l'assemblage
est le plus compact possible.
\index{compacité}\index{directionnelle, liaison}
Ces interactions sont responsables d'une température de
fusion très faible et des propriétés isolantes du composé
solide.
\index{fusion, température de!cristal de Van der Waals}
\index{conductivité électrique!cristal de Van der Waals}
\index{cristal!moléculaire|)}


\printindex
\end{document}
