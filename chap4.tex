\chapter{Cristaux covalents et moléculaires}
Le programme officiel ne demande que de <<relier les caractéristiques des liaisons
covalentes, des interactions de Van der Waals et des liaisons hydrogènes et les
\index{Van der Waals}
propriétés macroscopiques des solides correspondants>>. Quatre exemples seront étudiés
brièvement dans le cadre du programme.

\section{Exemple de structure covalente : le diamant \index{diamant|(}}
\subsection{Description de la maille}
Les atomes de carbone constituent une structure cubique à faces centrées avec
occupation d’un site tétraédrique sur deux par un atome de carbone (cf blende). Chaque
\index{blende}\index{carbone}
atome de carbone a un environnement tétraédrique de quatre carbones : la distance
entre deux carbones voisins est $d_\text{C-C} = 154$pm.

\begin{figure}
    \centering
    \includegraphics[width=5cm]{chap4/diamant.png}
    \caption{Vue éclatée d'une maille de diamant}\label{fig:4_diamant_eclatee}
\end{figure}

\subsection{Caractéristiques de la maille}
\subsubsection{Nombre de motifs par maille}
Strucuture cfc : motifs aux sommets et au centre des faces, avec occupation de la
moitié des 8 sites tétraédriques.
\begin{equation}
    Z = 8 \times \frac{1}{8} + 6 \times \frac{1}{2} + 4 \times 1 = 8
\end{equation}

\subsubsection{Coordinence}
La coordinence C/C est de 4.

\subsubsection{Paramètre géométrique}
Les atomes de carbones les plus proches sont selon la diagonale d’un cube d’arête
$\frac{a}{2}$ on obtient :
\begin{equation}
    \frac{a\sqrt{3}}{4} = d_\text{C-C}
\end{equation}
D'où $a = 356$pm.

\subsubsection{Compacité}
\begin{equation}
    C = \frac{8(\frac{4}{3}\pi r^3)}{a^3} = \np{0,34}
\end{equation}
C'est une structure peu compacte.


\subsection{Propriétés physiques}
Le carbone (Z = 6) possède 4 électrons de valence sur sa couche externe. Pour\index{valence}
satisfaire à la règle de l’octet, il est tétravalent. La tétravalence est réspectée dans la structure
\index{octet, règle de}
du diamant : chaque atome de carbone échange avec ses quatre voisins quatre liaisons
de covalence. Ces liaisons assurent une grande stabilité au diamant : il en résulte que
le diamant est isolant, rigide et dur par suite de la forte intéraction entre les atomes de
carbone.\index{conductivité électrique!diamant}\index{diamant!conductivité électrique}
Son énergie \index{cohésion, énergie de!diamant} \index{diamant!cohésion, énergie de}
de cohésion est de 717kJ/mol, sa température\index{fusion, température de!diamant} 
\index{diamant!fusion, température de}de fusion est 3500
\degre C.\index{propriétés mécaniques!diamant}
\index{diamant!propriétés mécaniques}\index{diamant|)}


\section{Autre structure covalente: le graphite}\index{graphite|(}
\subsection{Description de la maille}
Le graphite donne une structure en feuillets. Les atomes de carbone de type trigonal
\index{feuillet}\index{HC}
forment une structure hexagonale plane et régulière. Dans ce plan, les atomes voisins
sont équidistants (142pm). Les plans sont parallèles, distants de 335pm. Le passage
d’un plan A à un plan B se fait par translation de 141pm du plan A suivant un côté de
l’hexagone. Une translation de $-$141pm appliquée à B permet de retrouver le plan A.
Les plans se succèdent AB, AB, \dots
\begin{figure}
    \centering
    \includegraphics[width=8cm]{chap4/graphite_3d.png}
    \caption{Plans du graphite}\label{fig:4_graphite_3d}
\end{figure}

\subsection{Caractéristiques de la maille}
\subsubsection{Nombre de motifs par maille}
\begin{itemize}
    \item 8 sommets communs à 8 mailles
    \item 2 atomes situés sur les faces
    \item 4 atomes sur les arêtes
    \item 1 atome à l'intérieur
\end{itemize}
\begin{equation}
    Z = 8 \times \frac{1}{8} + 2 \times \frac{1}{2} + 4 \times
    \frac{1}{4} + 1 \times 1 = 4
\end{equation}
\subsubsection{Coordinence}
Chaque atome de carbone est trigonal, la coordinence
est donc de 3.
\subsubsection{Paramètre géométrique}
La distance entre deux atomes de carbones est $d_\text{C-C} = 142$pm.
Le rayon atomique vaut $r = 71$pm

\subsection{Propriétés physiques}
Dans un plan donnée, chaque atome de carbone ne forme que trois
liaisons de covalence avec trois atomes de carbone voisins.
Les différents feuillets interagissent par des forces de Van der
Waals faisant intervenir l'interaction des électrons délocalisés
\index{Van der Waals}\index{feuillet}
d'un feuillet avec les noyaux des atomes de carbone des feuillets
les plus proches.\\
Il en résulte que le graphite est conducteur et mou (il se fend
facilement le long de ses feuillets).\index{conductivité électrique!graphite}
\index{propriétés mécaniques!graphite}\index{graphite!propriétés mécaniques}
Cette structure en feuillets permet aux plans de glisser les uns
par rapport aux autres ce qui explique les propriétés lubrificatrices
du graphite (huiles graphitiques). C'est aussi un cristal anisotrope,
la conduction électrique est 200 fois plus élevée dans une direction
parallèle à un feuillet que dans la direction orthogonale.\index{anisotropie}
\index{graphite|)}


\section{Structure moléculaire : la glace}\index{glace|(}
\subsection*{Cristal moléculaire}\index{cristal!moléculaire}
Un cristal moléculaire résulte de la juxtaposition de molécules
qui gardent leur identité dans le cristal (comme à l'état gazeux).
De ce fait, les liaisons covalentes interatomiques au sein des
molécules restent pratiquement inchangées par rapport à l'état
gazeux.\index{gaz}\\
On distingue plusieurs types de cristaux moléculaires selon la
nature de l'interaction qui en assure la cohésion :
forces de Van der Waals ou liaison à hydrogène.
\index{Van der Waals}

\begin{rem}[Liaisons hydrogène]\index{liaison hydrogène}
    De 10 à 30kJ/mol : force intermoléculaire impliquant un
    \index{cohésion, énergie de!liaison hydrogène}
    atome d'hydrogène et un atome électronégatif.
    \index{électronégativité}
\end{rem}
\begin{figure}
    \centering
    \includegraphics[width=4cm]{chap4/liaison_h.png}
    \caption[Liaison hydrogène]{Représentation d'une liaison
        hydrogène (en bleu)}\label{fig:4_liason_h}
\end{figure}
\begin{rem}[Forces de Van der Waals]\index{Van der Waals}
    Nommées en l'honneur du physicien néerlandais Johannes
    Diderik van der Waals (1837\---1923) prix nobel de
    physique en 1910. Il fut le premier à introduire les
    effets de ces forces dans les équations d'état des gaz
    en 1873
\end{rem}
\begin{figure}
    \centering
    \includegraphics[width=9cm]{glace.png}
    \caption{Structure cristalline de l'eau}\label{fig:4_glace}
\end{figure}

\subsection{Description de la structure}
La structure de la glace est semblable à celle du diamant.\\
\index{diamant}
La glace se présente selon les conditions de température et
de pression sous plusieurs variétés allotropiques.
\index{allotropique, variété}
La variété stable dans les conditions ordinaires est
de type hexagonal. La variété de glace cubique s’observe
sous de très faibles pressions dans l’intervalle de
température 148\--188K.


Dans la glace de type hexagonal, les atomes d’oxygène
ont une structure diamant: occupation des nœuds d’un
réseau cfc, avec occupation en alternance d’un\index{CFC}
site tétrédrique sur deux.\index{hydrogène}\index{oxygène}
Les atomes d’hydrogène sont situés sur les segments
joignant deux atomes d’oxygène plus proches voisins,
mais à distance $d_1=98$pm d’un atome d’oxygène \-- selon
une liaison covalente\-- et à distance $d_2= 177$pm
de l’autre atome selon une liaison hydrogène.\index{liaison hydrogène}
Voir figure~\ref{fig:4_tétraèdre_glace} et
figure~\ref{fig:4_glace} où les liaisons hydrogène sont
représentées par les traits entre un atome d'hydrogène
et un atome d'oxygène.
\begin{figure}
    \centering
    \includegraphics[width=6cm]{chap4/glace.png}
    \caption{Détail de la maille de glace}
    \label{fig:4_tétraèdre_glace}
\end{figure}

\subsection{Cohésion}
Dans la glace, les liaisons dirigées résultent de liaisons
\index{directionnelle, liaison}
hydrogène. Il en résulte une structure lacunaire qui permet
la dissolution de composés gazeux et la formation
\index{solubilité}d’<<hydrates de gaz>>.

\subsection{Paramètre géométrique}
On notera :
\begin{description}
    \item[$a$] arête du cube de la maille
    \item[$d_\text{O-O}$] la distance entre deux atomes
        d'oxygène
    \item[$d_1$] quelque chose
    \item[$d_2$] autre chose
    \item[$\rho$] la masse volumique
    \item[$\mathcal{N}_A$] le nombre d'Avogadro
    \item[$M_{\text{H}_2\text{O}}$] la masse molaire de
        l'eau
\end{description}
On a :
\begin{equation}
    d_\text{O-O} = \frac{a\sqrt{3}}{4} = d_1 + d_2
    = 275\text{pm}
\end{equation}

\subsection{Masse volumique}
\begin{equation}
    \rho = \frac{8M_{\text{H}_2\text{O}}}{\mathcal{N}_A
        a^3} = \np{9,33}.10^2\text{kg.m}^{-3}
\end{equation}
\begin{rem}
    L'eau est une des rares substances pour laquelle la
    masse volumique du solide est inférieure à celle du
    liquide.\index{masse volumique}
\end{rem}
\index{glace|)}


\section{Cristaux moléculaires de Van der Waals}\index{Van der Waals}
\index{cristal!Van der Waals, de}
Les gaz nobles, constitués de molécules monoatomiques
\index{gaz noble}
cristallisent, sauf l’hélium, dans la structure cfc.
Le diiode, le dibrome, le dioxyde de carbone, ont
une structure dérivée de la structure cfc du\index{CFC}
fait de molécules non sphériques.\index{sphères dures, modèle des}
\index{hélium}\index{dibrome}\index{dioxyde de carbone}
\subsection{Cohésion}
La cohésion est assurée par des interactions de
Van der Waals. Celles ci sont de faible énergie:
10 à 100 fois moindre que celle des liaisons covalentes.
\index{cohésion, énergie de!cristal de Van der Waals}
Elles sont son dirigées, par conséquent l'assemblage
est le plus compact possible.\\
\index{compacité}\index{directionnelle, liaison}
Ces interactions sont responsables d'une température de
fusion très faible et des propriétés isolantes du composé
solide.
\index{fusion, température de!cristal de Van der Waals}
\index{conductivité électrique!cristal de Van der Waals}
